%&latex
% Copyright 2011 Ruslan Kiyanchuk (c) <ruslan.kiyanchuk@gmail.com>
% 
%/

\Chapter{Conclusions}
\label{sec:conclusions}

The accomplished analysis of cryptographic properties revealed the line of
symmetric ciphers current development. A global shift to portable devices usage
spawned a demand for very efficient and also secure cryptographic primitives.
Current usage of insecure ciphers in various fields of information technologies
proves the importance of carefull ciphers security evaluation before
deploying them into real systems. 

Standards for mobile communication tend to improve security and switch to
cryptographically strong ciphers in the near future as shown by 
analysis of ZUC stream cipher. Evaluation of the cipher allowed to outline the
best security decisions during the design of the cryptoalgorithm:
\begin{itemize}
    \item combination of XOR and modular addition helps to avoid possible side
        effects of each transformation; however it does not handle the case of
        zero addition, so introducing operations with constants would be an
        extra security precaution;
    \item exchanging half-words of the finite state machine registers increases
        its period and makes the behaviour unpredictable;
    \item combining the FSM output with LFSR words decreases the correlation of
        starting keystream bits with initial cipher state.
\end{itemize}

Efficiency and implementation compactness comparison of block ciphers indicated
that even though specifically designed for lightweight
purposes the PRESENT cipher is not better than the well researched GOST~28147-89
algorithm which can be implemented with even fewer gate equivalents. According
to the algebraic analysis of GOST replacing 8 distinct S-boxes with a single
one and avoid reversing subkeys during the last 8 rounds would not weaken the
cipher security, but would make it's implementation more compact making it the
most lightweight symmetric block cipher.

An algebraic equation system describing full GOST~28147-89 cipher contains
$10432$ polynomials in $4416$ variables. The polynomial system for the PRESENT
cipher that is designed for lightweight cryptography purposes contains $11067$
quadratic equations in $4216$ variables which is
slightly more than GOST. However PRESENT has much smaller key space and
requires more space for hardware implementation. It is claimed that AES cipher can be described with an algebraic system of
$8000$ quadratic equations in $1600$ unknowns which is signivicantly smaller
than the GOST equation system. 

As described in section \ref{sec:key-rec}, it is possible to solve a 5 round
GOST polynomial system with 4 pairs of plaintexts and ciphertexts at the
moment. Thereby the reduced GOST algorithm using 160 out of 256 key bits is
broken by an algebraic attack. Such statistics strengthens the opinion about
AES vulnerability to algebraic attacks. Solving the GOST equation system for
additional rounds requires more computation power, however finding the solution
may be possible on better computers. 

\labourprotection{
In chapter~\ref{sec:labour} the employee working conditions are analysed for their
compliance with normative documents on safety engineering and sanitaion. 
Harmful and dangerous production factors are retrieved and evaluated using the
built ``Human--Machine--Environment'' interaction system. Correspondent safety
measures are developed in order to provide favourable working conditions.
}


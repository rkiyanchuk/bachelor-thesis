%&latex
% Copyright 2011 Zoresvit (c) <zoresvit@gmail.com>
%
%
%/

\newpage
\selectlanguage{ukrainian}
\chapter*{РЕФЕРАТ}
    Бакалаврська робота містить \total{page}~сторінки,
    \totalfigures~рисунків,
    \totaltables~таблиць, 4~додатки та 74~джерела.

    У роботі представлено аналіз перспективних симетричних шифрів, що проходять процес
    міжнародної стандартизації.

    Розглянуто криптографічні властивості поточного шифру ZUC, перспективного
    для використання у новітньому стандарті мобільного зв'язку LTE. Виявлено
    небажану властивість лінійного перетворення,
    а також вдалі з точки зору криптографічної стійкості рішення, що
    запобігають використанню даної властивості для здійснення атаки на шифр.

    Досліджено придатність і можливі модифікації шифрів PRESENT та
    ГОСТ~28147-89 для застосування в області мало-ресурсної (lightweight)
    криптографії. Виявлено, що криптоалгоритм ГОСТ має високу продуктивність та
    компактну реалізацію, а по деяким параметрам випереджає PRESENT, який
    спеціально розроблений для використання на пристроях з обмеженими
    ресурсами.

    Далі досліджується криптографічна стійкість ГОСТ~28147-89 до алгебраїчного
    криптоаналізу. Зменшена версія криптоалгоритму ГОСТ, що складається з 5
    раундів, виявилася вразливою до алгебраїчної атаки з використанням
    додатку CryptoMiniSat для вирішення системи нелінійних рівнянь та
    відновлення всіх підключів зашифрування протягом хвилин. \\[1em]

    СИМЕТРИЧНІ ШИФРИ, АЛГЕБРАЇЧНИЙ КРИПТОАНАЛІЗ, ЛІНІЙНЕ ПЕРЕТВОРЕННЯ,
    ГОСТ~28147, PRESENT, ZUC.

\selectlanguage{english}
\newpage
\chapter*{ABSTRACT}
This thesis contains \total{page}~pages, \totalfigures~figures, \totaltables~tables,
    4~appendices and 74~references.

    Analysis of perspective symmetric ciphers that are
    undergoing the process of international standardization is presented.

    Cryptographic properties of ZUC, a perspective stream cipher considered
    for use in evolving LTE standard, are evaluated. Negligible defect in its
    linear transformation has been revealed together with some good design
    decisions that prevent cryptanalyst from exploiting it and attacking the cipher.

    Suitability and possible modifications for lightweight cryptography purposes of
    PRESENT and GOST~28147-89 ciphers are researched. The GOST cryptoalgorithm is
    shown to have high performance and compact implementation while some of its
    parameters are better than those of PRESENT which is designed for low resources
    hardware.

    Further the strength of GOST~28147 cipher to algebraic cryptanalysis is
    researched. Reduced version of 5 rounds GOST cryptoalgorithm is found to be
    vulnerable to an algebraic attack using CryptoMiniSat for solving multivariate
    equation system and recovering all used encryption subkeys within minutes. \\[1em]

    SYMMETRIC CIPHERS, ALGEBRAIC CRYPTANALYSIS, LINEAR TRANSFORMATION, GOST~28147,
    PRESENT, ZUC.
\newpage

%&latex
% Copyright 2011 Ruslan Kiyanchuk (c) <ruslan.kiyanchuk@gmail.com>
% 
% 
%/

\chapter{ALGEBRAIC ANALYSIS OF GOST~28147-89}
\label{sec:algebraic}

\section{Methods for solving equation systems}

Being a fairly new technique, algebraic cryptanalysis is one of the most
promissory and powerful method for analysing cryptographic
algorithms~\cite{Albrecht2010}. It implies modeling a cryptographic algorithm
by a set of algebraic equations that form a multivariate polynomial equation
system over finite field. The lower degree such system has the easier it is to
solve, so it is a rule of thumb in algebraic analysis to construct algebraic
sytems that contain only quadratic multivariate (MQ)\footnote{It also became a
widespread practice to denote the problem of solving multivariate quadratic
equation systems as MQ for short.} equations.

The essence of algebraic cryptanalysis is an assumption made by Claude Shannon
in~\cite{shannon:secrecy} that binds cipher security to the difficulty of
solving the corresponding algebraic equations set:
``Breaking a good cipher should require as much work as solving a system of
simultaneous equations in a large number of unknowns of a complex
type''.

The main advantage of algebraic cryptanalysis over other methods is the need
for only few pairs of plaintexts and ciphertexts. Breaking Keeloq cipher is
a good example of successful algebraic attack on full scale
cryptoalgorithm~\cite{bard2009algebraic}. The attack allows to recover the
encryption key $2^{14.77}$ times faster than a brute force search.

An Advanced Encryption Standard (AES) also widely used outside of the USA is
potentially vulnerable to XSL attack (a type of algebraic cryptanalysis).
It's complexity claimed to be $2^{100}$ significantly weakens the cipher. Even
though the practical applicability of the attack hasn't yet been proven, the
algebraic AES structure compliance to such analysis.


Given an equation system that completely describes the cryptographic algorithm
one gets a powerful tool for researching its hidden properties. However such
systems are hard to solve for full scale ciphers. A complexity of polynomial
systems is usually described by the number of equations and the number of
variables it contains~\cite{bard2009algebraic}. 

It is shown in~\cite{Courtois:MQ-NPhard} that solving equation systems is
NP-hard, but it's complexity significantly drops if the system becomes
overdefined, that is there are much more equations than
unknowns~\cite{DBLP:conf/asiacrypt/CourtoisP02}.

\subsection{Gr\"obner basis}

Finding a Gr\"obner basis is equivalent to solving various problems concerning
polynomial systems~\cite{bard2009algebraic}. As Gaussian elimination method
solves linear equation systems, the Gr\"obner basis is designed to do the same
for non-linear polynomial systems.

In~\cite{Albrecht2006} the Gr\"obner basis is defined as following.

    For a fixed monomial order a finite subset $G = \{g_0, \hdots, g_{m-1}\}$
    of an ideal $\mathbb{I}$ is said to be a Gr\"obner basis if
    \begin{equation}
    \label{eqn:groebner}
    \left< LT(g_0), \hdots, LT(g_{m-1}) \right> = \left< LT(I) \right> \enspace,
    \end{equation}
    where $LT$ denotes the leading term of a polynomial.

Notably the leading term of every polynomial from $\mathbb{I}$ is divisible by
the leading term of at least one polynomial from $\mathbb{G}$.

Gr\"obner basis has been successfully applied for attacking several ciphers,
including FLURRY and CURRY~\cite{Pyshkin2008:groebner} and even showed to be
more efficient than SAT solvers in algebraic attack on
Bivium~\cite{springerlink:10.1007/s11786-009-0016-7}.

\subsection{SAT solvers}

Another approach to solving MQ problem is SAT\footnote{SAT is an abbreviation
denoting boolean satisfiability problem.} solvers that are used for
determining a set of variables that would satisfy a given boolean formula. 

An efficient method of converting equation systems from the algebraic normal
form (ANF) to the conjunctive normal form (CNF) was proposed
in~\cite{cryptoeprint:2007:024}. After conversion the resulting system is
solved by a SAT solver of cryptanalyst's choice. SAT solvers also require less
memory and make it possible to solve problems infeasible for Gr\"obner basis
algorithms.

SAT solvers gained lots of attention lately since SAT problem is
\mbox{NP-complete} so
developing an SAT solving algorithm running in polynomial time would mean that
$P = NP$. So proving (or disapproving) the equivalence of these two complexity
classes would affect not only asymmetric ciphers based on factorization
problem, but any ciphers that may be efficiently described by an algebraic
equation system.
 
\section{Description of GOST~28147-89 cipher}
\label{sec:algebraic-gost}

Officially adopted in 1989 the GOST~28147 cipher has been developed in
former USSR and is now the encryption standard in most CIS countries.
For more than 20 years of cryptanalysis no efficient attack that would
significantly reduce the cipher security had been found. 

GOST~28147-89 is a symmetric block cipher with a key length of 256 bits. It's
represented by a Feistel network of 32 rounds. The round function consists of
key addition modulo $2^{32}$, substitution layer represented by eight 4-bit
S-boxes, and a cyclic left shift by 11 bits (figure~\ref{fig:gost-round-func}). S-boxes are not defined in the 
original standard~\cite{GOST28147}. For some time they were considered to be
another secret parameter, but such approach caused more problems than gained
additional security. Use of different S-boxes set caused some cipher
implementations to be incompatible. Later all possible benefits of secret
substitution layer were scattered by introducing a method to recover all
unknown S-boxes in $2^{32}$ encryptions~\cite{saarinen1998:sboxes}.
\begin{figure}[htbp]
	\centering
    \input{images/gost}
	\caption{GOST~28147-89 round function}
	\label{fig:gost-round-func}
\end{figure}
The 256-bit key is split into 8 32-bit subkeys that are used sequentially
throughout the Feistel network. During the last 8 rounds subkeys are fed in
reverse order. Half blocks are not switched after the last round in order to
make encryption and decryption procedures similar.

 However some recent works 
\cite{cryptoeprint:2011:626, Courtois:cryptoeprint_2011} claim the fastest attack on
GOST has complexity $2^{224}$ and requires $2^{32}$ known plaintexts. 

\section{Constructing equation systems for GOST~28147-89}

Due to simple and compact structure of the GOST cipher there are only two
transformation whose equations description is nontrivial: modular addition and
S-boxes.

Consider variable format $X_{r, \, b}$, where $r$ denotes round for which the
variable is defined and $b$ is a bit number of the block. Then defining one
round of GOST cipher requires 4 sets of variables: 
\begin{enumerate}
    \item $X_{r, \, 0 \hdots 63}$ for input block, 
    \item $K_{r \% 8, \, 0 \hdots 31}$ for a subkey (where $\%$
        denotes modulo reduction), 
    \item $Y_{r, \, 0 \hdots 31}$ for the result of key addition, 
    \item $Z_{r, \, 0 \hdots 31}$ for the result of S-box substitution. 
\end{enumerate}

Input to a key addition modulo $2^{32}$ are variables 
$X_{r, \, 31 \hdots 63}$ of the right input half-block and 
$K_{r \% 8, \, 0 \hdots 31}$ of the subkey. The result of key injection is
assigned to variables $Y_{r, \, 0 \hdots 31}$. GOST~28147-89 key injection
transformation can be described with $93$ equations. Obtaining the equations
for modular addition is described further in section~\ref{seq:key-add-eqn}.

An arbitrary S-box is fully defined by equations of degree 2 that can be
obtained by finding null space equations as described in~\cite{kleiman:xsl}.
During this research the S-boxes used in GOST cipher implementation are those
proposed in \cite{GOST3411} and given in appendix~\ref{app:gost-sboxes}. Every
such S-box is defined by $21$ equations each containing up to 14 monomials, so
the total number of equations for S-box layer is $168$.  Input variables $Y_{r,
\, 0 \hdots 31}$ are fed to substitution layer resulting to 32 output variables
$Z_{r, \, 0 \hdots 31}$.

Cyclic shift is equivalent to renaming variables $Z_{r, \, 0 \hdots 31}$ to 
$Z_{r, \, \{21 \hdots 31, 0 \hdots 20\}}$. Equations for XORing round function
output with the left input half-block and switching the half-blocks are
trivial. The resulting polynomials of one Feistel network round are then
assigned to next round variables $X_{r+1, \, 0 \hdots 63}$.

\subsection{Constructing equations for modular addition}
\label{seq:key-add-eqn}

During the Ukrainian national public cryptographic competition the following
method for defining modular addition equation system has been proposed by the
author of Labyrinth block cipher.

Modular addition $R = X + Y$ of two $n$-bit numbers is defined as 
\begin{equation}
\label{eqn:add}
X = (x_0, \hdots, x_{n-1}), \; 
Y = (y_0, \hdots, y_{n-1}), \;
R = (r_0, \hdots, r_{n-1}),
\end{equation}
where $i$ --- is a bit number, so $x_i$ represents $i$-th bit of number $X$. 
Then the addition on a bit level is defined as follows:
\begin{equation}
\label{eqn:trivial-mod-add}
r_i = x_i \oplus y_i \oplus c_{i-1} \enspace,
\end{equation}
where $c_i$ is a carry bit variable and is defined by \eqref{eqn:carry-bit}.
\begin{equation}
\label{eqn:carry-bit}
c_i = r_{i+1} \oplus x_{i+1} \oplus y_{i+1} \enspace.
\end{equation}
The goal is to get null space equations for every addition bit without
explicitly using the carry variable. It turns out that for addition bits 
$0 < i < (n - 1)$ three implicit equations may be defined:
\begin{equation}
\label{eqn:add-with-carry}
\left\{
	\begin{array}{ll}
        (x_i \oplus r_i) (x_i \oplus c_i) = 0; \\
        (y_i \oplus r_i) (y_i \oplus c_i) = 0; \\
        (x_i \oplus y_i) \cdot r_i \oplus x_i y_i \oplus x_i \oplus y_i \oplus c_i = 0.
	\end{array} \right.
\end{equation}
The final equation system that defines every addition bit may be extracted
from~\eqref{eqn:add-with-carry} by substituting every $c_i$ variable according
to~\ref{eqn:carry-bit}:
\begin{equation}
\label{eqn:mod-add}
\left\{
	\begin{array}{ll}
        x_i \oplus x_i r_i \oplus x_i r_{i+1} \oplus x_i x_{i+1} \oplus x_i y_{i+1} \oplus r_i r_{i+1} \oplus r_i x_{i+1} \oplus r_i y_{i+1} = 0; \\
        y_i \oplus y_i r_i \oplus y_i r_{i+1} \oplus y_i x_{i+1} \oplus y_i y_{i+1} \oplus r_i r_{i+1} \oplus r_i x_{i+1} \oplus r_i y_{i+1} = 0; \\
        x_i r_i \oplus y_i r_i \oplus x_i y_i \oplus x_i \oplus y_i \oplus r_{i+1} \oplus x_{i+1} \oplus y_{i+1} = 0.
	\end{array} \right.
\end{equation}
Thereby a single addition bit is defined by three quadratic equations each
containing 12 terms of degree 2. Even though the equations in
\eqref{eqn:mod-add} fully describe $n$-bit modular addition, adding an
additional equation $r_0 = x_0 + y_0$ for the very first bit was found to have
crucial influence on finding system solutions.

The chosen approach of constructing polynomial system for the GOST cipher
allows to define a single cipher round with 325 quadratic equations. An
algebraic equation system describing full GOST~28147-89 cipher contains $10432$
polynomials in $4416$ variables. In fact in case of using the subkeys in straight
order (without reversing during last 8 rounds) the number of polynomials and
variables in the system stays the same. That means the reversing the subkeys
during last 8 rounds does not strengthen the cipher but only complicates its
implementation. 

\section{Key recovery for 5 rounds}
\label{sec:key-rec}

In order to solve the equation system some chosen plaintext and the
corresponding ciphertext are injected into the system (by substituting
variables of the first and the last block to their corresponding values).
Since GOST~28147-89 cipher has the key length of 256 bits, having an equation
system for a single pair of plaintext and ciphertext doesn't allow to recover
the encryption key. The solution combines several equation systems for
distinct plaintexts and ciphertexts in order to inject more information into
the resulting polynomial system. Considering the 64-bit input block it is
evident that a successful 256-bit key recovery should be possible using an
equation system for at least 4 plaintext/ciphertext pairs.

Using a premier SAT solver~\cite{soos:cryptominisat} 5 rounds of GOST~28147-89
cipher had been broken and a complete list of the used subkeys completely recovered. After
observations during the research two factors were found to influence the
equation system solving complexity. The more zeroes encryption key contains,
the easier the polynomial system is for solving since modular addition of zero
does not generate carries. Also the chosen plaintexts should have maximum
hamming distance in order to introduce enough information to the system.

Finding solution for more rounds of GOST~28147-89 polynomial system requires
much larger time gap on an ordinary laptop. However the memory requirements are
small so taking into account the simplicity and uniformity of the cipher
structure it is rational to assume that a more powerful computer would allow to
break more rounds of the cipher.

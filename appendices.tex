%&latex
% Copyright 2011 Ruslan Kiyanchuk (c) <ruslan.kiyanchuk@gmail.com>
% 
% 
%/

\titleformat{\chapter}[display]{\center\normalfont\normalsize}{\chaptertitlename\hspace{1em}\thechapter}{1pc}{\MakeUppercase}[\thispagestyle{myheadings}]

\begin{appendices}


\chapter{SOURCE CODE FOR \mbox{GOST~28147-89} POLYNOMIAL SYSTEM GENERATOR}

\begin{lstlisting}
from copy import deepcopy
from sage.crypto.mq.sbox import SBox
from sage.rings.polynomial.multi_polynomial_sequence import PolynomialSequence
from sage.rings.polynomial.multi_polynomial_sequence import PolynomialSequence_generic


def inject(self, vars, values):
    sub_values = dict(zip(vars, values))
    return self.subs(sub_values)
PolynomialSequence_generic.inject = inject

def join_systems(mqsystems, instances):
    var_names = flatten([i.ring.variable_names() for i in instances])
    common_vars = list(set(var_names))
    common_ring = BooleanPolynomialRing(len(common_vars), common_vars, \
                                        order='degrevlex')
    new_mqsystem = PolynomialSequence([], common_ring)
    for s in mqsystems:
        new_mqsystem.extend(list(s))
    return new_mqsystem


class Gost:
    def _varformatstr(self, name):
        l = str(max([len(str(self.nrounds)), len(str(self.block_size - 1))]))
        return name + "%0" + l + "d" + "%0" + l + "d"

    def _varstrs(self, name, round):
        s = self._varformatstr(name)
        if s.startswith(self.var_names['block']):
            return [s % (round, i) for i in range(self.block_size)]
        else:
            return [s % (round, i) for i in range(self.halfblock_size)]

    def gen_vars(self, name, round_):
        return [self.ring(e) for e in self._varstrs(name, round_)]

    def int2bits(self, num, bits):
        num = Integer(num)
        return num.digits(base=2, padto=Integer(bits))

    def bits2int(self, num_bits):
        num = ''.join([str(i) for i in reversed(num_bits)])
        return int(num, 2)

    def __init__(self, **kwargs):
        self.SBOX_SIZE = 4
        self.nrounds = kwargs.get('rounds', 32)
        if self.nrounds < 8:
            self.key_length = self.nrounds
        else:
            if self.nrounds % 8 != 0:
                raise ValueError('Number of rounds must be multiple of 8')
            self.key_length = 8
        self.block_size = kwargs.get('block_size', 64)
        if self.block_size % self.SBOX_SIZE != 0 or self.block_size < 8:
            raise ValueError('Block size must be multiple of 4 
                    (due to SBox) and greater than 8 (due S-box size)')
        self.halfblock_size = self.block_size / 2
        self.key_order = kwargs.get('key_order', 'frwrev')
        if self.key_order not in ['frw', 'frwrev']:
            raise ValueError('Unsupported key ordering')
        if self.key_order is 'frwrev' and self.nrounds % 8 != 0:
            raise ValueError('frwrev key ordering is only possible 
                    for nrounds to be multiple of 8')
        self.key_add = kwargs.get('key_add', 'mod')
        if self.key_add not in ['mod', 'xor']:
            raise ValueError('key_add may be set to `mod` or `xor`')

        self._init_sboxes(kwargs.get('sboxes', None))
        pre = kwargs.get('prefix', '')
        self.var_names = {'key': 'K', 
                'block': pre + 'X', 
                'sum': pre + 'Y',
                'sbox': pre + 'Z'}
        self.gen_ring()

    def _init_sboxes(self, sboxes):
        self._default_sboxes = [
                [4, 10, 9, 2, 13, 8, 0, 14, 6, 11, 1, 12, 7, 15, 5, 3],
                [14, 11, 4, 12, 6, 13, 15, 10, 2, 3, 8, 1, 0, 7, 5, 9],
                [5, 8, 1, 13, 10, 3, 4, 2, 14, 15, 12, 7, 6, 0, 9, 11],
                [7, 13, 10, 1, 0, 8, 9, 15, 14, 4, 6, 12, 11, 2, 5, 3],
                [6, 12, 7, 1, 5, 15, 13, 8, 4, 10, 9, 14, 0, 3, 11, 2],
                [4, 11, 10, 0, 7, 2, 1, 13, 3, 6, 8, 5, 9, 12, 15, 14],
                [13, 11, 4, 1, 3, 15, 5, 9, 0, 10, 14, 7, 6, 8, 2, 12],
                [1, 15, 13, 0, 5, 7, 10, 4, 9, 2, 3, 14, 6, 11, 8, 12]
                ]
        if not sboxes:
            sboxes = self._default_sboxes
        else:
            for s in sboxes:
                if len(s) != 2 ^ self.SBOX_SIZE: 
                    raise TypeError('S-box must be 4x4 bits (0..15)')
        self.sboxes = [SBox(i, big_endian=False) for i in sboxes]

    def gen_ring(self):
        nr = self.nrounds
        bs = self.block_size
        hbs = self.halfblock_size
        var_names = []
        halfblock_vars = [self.var_names['key'], \
                            self.var_names['sum'], \
                            self.var_names['sbox']]
        for r in range(nr):
            var_names += [self._varformatstr(v) % (r, b) 
                    for v in halfblock_vars for b in xrange(hbs)]
        for r in range(nr + 1):
            var_names += [self._varformatstr(self.var_names['block']) % (r, b) 
                    for b in xrange(bs)]
        self.ring = BooleanPolynomialRing(len(var_names), var_names, \
                                            order='degrevlex') 
        return self.ring

    def polynomial_system(self):
        hbs = self.halfblock_size
        mqsystem_parts = []

        kvars = list()
        for i in range(self.key_length):
            kvars.append(map(self.ring, \
                    self.gen_vars(self.var_names['key'], i)))

        for i in range(self.nrounds):
            xvars = map(self.ring, self.gen_vars(self.var_names['block'], i))
            yvars = map(self.ring, self.gen_vars(self.var_names['sum'], i))
            zvars = map(self.ring, self.gen_vars(self.var_names['sbox'], i))
            next_xvars = map(self.ring, \
                    self.gen_vars(self.var_names['block'], i + 1))
            polynomials = []
            
            if self.key_order == 'frwrev' and \
                    i >= (self.nrounds - self.key_length):
                k = self.key_length - 1 - (i % self.key_length)
            else:
                k = i % self.key_length
            polynomials += self.add_round_key(xvars[:hbs], kvars[k], yvars)
            polynomials += self.polynomials_sbox(yvars, zvars)
            zvars = self.shift(zvars)
            xored = self.xor_blocks(xvars[hbs:], zvars)
            if i < self.nrounds - 1:
                # R_{i+1} = L_i xor f(R_i)
                polynomials += [x + y for x, y in zip(xored, \
                                                        next_xvars[:hbs])]
                # L_{i+1} = R_i
                polynomials += [x + y for x, y in zip(xvars[:hbs], \
                                                        next_xvars[hbs:])]
            else:
                # No block swapping in the last round.
                # R_{i+1} = L_i
                polynomials += [x + y for x, y in zip(xvars[:hbs], \
                                                        next_xvars[:hbs])]
                # L_{i+1} = L_i xor f(R_i)
                polynomials += [x + y for x, y in zip(xored, \
                                                        next_xvars[hbs:])]
            mqsystem_parts.append(polynomials)
        return PolynomialSequence(mqsystem_parts, self.ring)

    def polynomials_sbox(self, yvars, zvars):
        polynomials = list()
        sboxes = deepcopy(self.sboxes)
        for i in range(self.halfblock_size / self.SBOX_SIZE):
            nbit = i * self.SBOX_SIZE
            current_sbox = sboxes[i % len(sboxes)]
            pols = current_sbox.polynomials()
            gens = current_sbox.ring().gens()
            new_gens = yvars[nbit:nbit + self.SBOX_SIZE] + \
                    zvars[nbit:nbit + self.SBOX_SIZE] 
            sub = dict(zip(gens, new_gens))
            pols = [p.subs(sub) for p in pols]
            polynomials += pols
        return polynomials

    def __repr__(self):
        gost_id = 'GOST cipher '
        gost_id += '(Block Size = %d, Rounds = %d, '
        gost_id += 'Key Addition = %s, Key Order = %s)' 
        params = (self.block_size, self.nrounds, self.key_add, self.key_order)
        return gost_id % params

    def add_round_key(self, a, b, r=[]):
        hbs = self.halfblock_size
        is_defined  = lambda vals: all([p.constant() for p in vals])
        # If all variables are defined, just compute the result instead of
        # generating polynomials.
        if is_defined(a) and is_defined(b):
            if self.key_add is 'mod':
                data = self.bits2int(a)
                subkey = self.bits2int(b)
                modulo_mask = (1 << hbs) - 1
                result = (data + subkey) & modulo_mask
                return [self.ring(i) for i in self.int2bits(result, hbs)]
            if self.key_add is 'xor':
                return [x + y for x, y, in zip(a, b)]    
        else:
            if self.key_add is 'mod':
                pols = list()
                pols.append(a[0] + b[0] + r[0]) 
                for i in range(0, hbs - 1):
                    pols.append(a[i] + a[i] * r[i] + a[i] * r[i+1] + a[i] *
                            a[i+1] + a[i] * b[i+1] + r[i] * r[i+1] + r[i] *
                            a[i+1] + r[i] * b[i+1])
                    pols.append(b[i] + b[i] * r[i] + b[i] * r[i+1] + b[i] *
                            a[i+1] + b[i] * b[i+1] + r[i] * r[i+1] + r[i] *
                            a[i+1] + r[i] * b[i+1])
                    pols.append(a[i] * r[i] + b[i] * r[i] + a[i] * b[i] + a[i]
                            + b[i] + r[i+1] + a[i+1] + b[i+1])
                return pols
            if self.key_add is 'xor':
                return [x + y + z for x, y, z in zip(r, a, b)]    

    def shift(self, halfblock):
        shift = ceil(self.halfblock_size / 3)
        halfblock = halfblock[self.halfblock_size-shift:self.halfblock_size]+\ 
                halfblock[0:self.halfblock_size-shift]
        return halfblock

    def substitute(self, halfblock):
        result = []
        for i in range(self.halfblock_size / self.SBOX_SIZE):
            nbit = i * self.SBOX_SIZE
            plain = halfblock[nbit:nbit + self.SBOX_SIZE]
            sub = self.sboxes[i % len(self.sboxes)](plain)
            sub = [self.ring(j) for j in sub]
            result += sub
        return result

    def xor_blocks(self, left, right):
        return [x + y for x, y in zip(left, right)]

    def feistel_round(self, block, subkey):
        hbs = self.halfblock_size
        bs = self.block_size
        n1 = block[0:hbs] # left
        n2 = block[hbs:bs]; # right
        temp = n1
        n1 = self.add_round_key(n1, subkey)
        n1 = self.substitute(n1);
        n1 = self.shift(n1);
        n2 = self.xor_blocks(n1, n2)
        n1 = temp
        return n2 + n1

    def _cast_params(self, data_, key_):
        bs = self.block_size
        data = deepcopy(data_)
        key = deepcopy(key_)
        if not isinstance(data, list):
            data = self.int2bits(data, bs)
        data = [self.ring(i) for i in data]
        if len(key) != self.key_length:
            raise TypeError('Key should be of length ' + str(self.key_length))
        # coerse key bits to ring elements
        for i in range(len(key)):
            if not isinstance(key[i], list):
                key[i] = self.int2bits(key[i], self.halfblock_size)
            key[i] = [self.ring(j) for j in key[i]]
        return data, key

    def encrypt(self, data_, key_):
        hbs = self.halfblock_size
        bs = self.block_size
        if isinstance(data_, list):
            is_list = true;
        else:
            is_list= false;
        data, key = self._cast_params(data_, key_)

        for i in range(self.nrounds):
            if self.key_order == 'frwrev' and \
                    i >= (self.nrounds - self.key_length):
                k = self.key_length - 1 - (i % self.key_length)
            else:
                k = i % self.key_length
            data = self.feistel_round(data, key[k])
        data = data[hbs:bs] + data[0:hbs]
        if is_list:
            return data
        else:
            return self.bits2int(data)

    def decrypt(self, data_, key_):
        hbs = self.halfblock_size
        bs = self.block_size
        if isinstance(data_, list):
            is_list = true;
        else:
            is_list= false;
        data, key = self._cast_params(data_, key_)

        key.reverse()
        for i in range(self.nrounds):
            if self.key_order == 'frwrev' and i < self.key_length:
                k = self.key_length - 1 - (i % self.key_length)
            else:
                k = i % self.key_length
            data = self.feistel_round(data, key[k])
        data = data[hbs:bs] + data[0:hbs]
        if is_list:
            return data
        else:
            return self.bits2int(data)

    def random_key(self):
        key = [list(random_vector(int(self.halfblock_size), x=2)) 
                for _ in range(self.key_length)]
        key = [map(self.ring, i) for i in key]
        return key

    def random_block(self):
        return map(self.ring, list(random_vector(self.block_size, x=2)))

    def test_mqsystem(self, f_):
        if f_.ring() is not self.ring:
            raise TypeError('Tested MQ system has been generated by a ' 
                    'different GOST instance')
        f = deepcopy(f_)
        print 'Testing MQ system', f
        bs = self.block_size
        plaintext = self.random_block()
        key = self.random_key()
        ciphertext = self.int2bits(self.encrypt(plaintext, key), bs)
        f = f.subs(dict(zip(self.gen_vars(self.var_names['block'], \
                                            0), plaintext)))
        f = f.subs(dict(zip(self.gen_vars(self.var_names['block'], \
                                            self.nrounds), ciphertext)))
        for i in range(self.nrounds):
            k = i % self.key_length
            f = f.subs(dict(zip(self.gen_vars(self.var_names['key'], i), \
                                                                key[k])))
        s = f.ideal().interreduced_basis()
        if s == [1]:
            print 'MQ System for' + str(self) + 'is INCORRECT'
            print f
            return False
        else:
            return True


    def test_cipher(self):
        keys = [
                [0x0, 0x0, 0x0, 0x0, 0x0, 0x0, 0x0, 0x0], 
                [0xFFFFFFFF, 0xFFFFFFFF, 0xFFFFFFFF, 0xFFFFFFFF, \
                    0xFFFFFFFF, 0xFFFFFFFF, 0xFFFFFFFF, 0xFFFFFFFF], 
                [0x01234567, 0x89ABCDEF, 0x01234567, 0x89ABCDEF, \
                    0x01234567, 0x89ABCDEF, 0x01234567, 0x89ABCDEF], 
                [0x6ebabf8d, 0x1a8cad60, 0x124744f9, 0xd400b5d8, \
                    0xa721e3fd, 0x11d0702d, 0x06fd4827, 0x476df4bf]
                ]
        plain = [
                0x0000000000000000, 0xBDBDBDBDACACACAC, 0xFFFFFFFFFFFFFFFF, \
                    0x89ABCDEF01234567, 0xC0AE942BC8A99A39
                ]
        cipher = [
                0x12610BE2A6C2FDC9, 0xA587E5D3F6DFB6F4, 0x029BFE67A9364E44, \
                    0x523FC1A6AEC71B9A, 0x780CB7CE063F59E2,
                0x9057C2CF13AAAD6D, 0xF7B085BA4771F406, 0x780416781B29BC06, \
                    0xE596A183FD645558, 0x8EC42736538740AB,
                0xF1956B1D0A1A67DE, 0x9E4C808408DCDBDC, 0x7738AF92DE8FC770, \
                    0x9C44633FCEC0A03E, 0xC23013406002E268,
                0x91DB8E1FE489FAEF, 0x547BF353604A8190, 0x92B517E3CC91B9D0, \
                    0x1F5C2195762513E2, 0xE99D1C8DFF44A74C
                ]

        def show_failed(plain, key, result, expected):
            print 'GOST FAILED'
            print 'key:\t\t', ['0x%8.8X' % subk for subk in key]
            print 'plaintext:\t', "0x%16.16X" % plain
            print 'expected:\t', "0x%16.16X" % expected
            print 'actual:\t\t', "0x%16.16X" % result

        if self.block_size == 64 and self.nrounds == 32 and \
                self.key_add == 'mod' and self.key_order == 'frwrev':
            print 'Testing ', self, 'using test vectors...'
            it = cipher.__iter__()
            for k in keys:
                for p in plain:
                    c = self.encrypt(p, k)
                    expected = it.next() 
                    if c != expected:
                        show_failed(p, k, c, expected)
                        return False
        print 'Testing', self, 'by correct decryption...'
        x = randint(0, 2^self.block_size)
        key = list(random_vector(self.key_length, x = 2^self.block_size))
        c = self.encrypt(x, key)
        d = self.decrypt(c, key)
        if x != d:
            show_failed(x, key, d, x)
            return False
        return True
\end{lstlisting}


\chapter{SOURCE CODE FOR SOLVING 5 ROUND GOST EQUATION SYSTEM}
\label{app:solving-sat}

\begin{lstlisting}
attach gost.sage
attach anf2cnf.py

nr = 5
bs = 64
hbs = int(bs/2)
varnames = ['a', 'b', 'c', 'd']
gosts = [Gost(block_size=bs, 
    rounds=nr, 
    key_add='mod', 
    key_order='frw', 
    prefix=i) for i in varnames]

print 'constructing MQ systems...'
mqsystems = [i.polynomial_system() for i in gosts]

print 'generating plaintext/ciphertext data...'
inputs = [i.random_block() for i in gosts]
key = gosts[0].random_key()
outputs = [gosts[i].encrypt(inputs[i], key) for i in range(len(gosts))]

print  'injecting known variables...'
for i in range(len(mqsystems)):
    mqsystems[i] = mqsystems[i].inject(gosts[i].gen_vars(
            gosts[i].var_names['block'], 0), inputs[i])
    mqsystems[i] = mqsystems[i].inject(gosts[i].gen_vars(
            gosts[i].var_names['block'], nr), outputs[i])

print 'combining MQ systems...'
f = join_systems(mqsystems, gosts)

print 'solving MQ system with SAT solver...'
print time.ctime()
solver = ANFSatSolver(f.ring())
s, t = solver(f)
print 'DONE'
print time.ctime()

recovered_key = []
r = f.ring()
for i in range(len(key)):
    var_names = map(str, gosts[0].gen_vars(
            gosts[0].var_names['key'], i))
    var_names.sort()
    var_names = map(r, var_names)
    recovered_key.append([s[j] for j in var_names])

if gosts[0].int2bits(gosts[0].encrypt(
    inputs[0], recovered_key), bs) == outputs[0]:
    if key == recovered_key:
        print recovered_key
    else:
        print 'FOUND ANOTHER KEY'
        print 'actual'
        print key
        print 'found'
        print recovered_key
\end{lstlisting}

\chapter{S-BOXES USED IN GOST~28147-89 IMPLEMENTATION}
\label{app:gost-sboxes}

\begin{equation}
    \nonumber
    \begin{array}{lllllllllllllllll}
        S_1 =&  \{4, &  10, &  9, &  2, &  13, &  8, &  0, &  14, &  6, &  11, &  1, &  12, &  7, &  15, &  5, &  3 \} \\
        S_2 =&  \{14, &  11, &  4, &  12, &  6, &  13, &  15, &  10, &  2, &  3, &  8, &  1, &  0, &  7, &  5, &  9 \} \\
        S_3 =&  \{5, &  8, &  1, &  13, &  10, &  3, &  4, &  2, &  14, &  15, &  12, &  7, &  6, &  0, &  9, &  11 \} \\
        S_4 =&  \{7, &  13, &  10, &  1, &  0, &  8, &  9, &  15, &  14, &  4, &  6, &  12, &  11, &  2, &  5, &  3 \} \\
        S_5 =&  \{6, &  12, &  7, &  1, &  5, &  15, &  13, &  8, &  4, &  10, &  9, &  14, &  0, &  3, &  11, &  2 \} \\
        S_6 =&  \{4, &  11, &  10, &  0, &  7, &  2, &  1, &  13, &  3, &  6, &  8, &  5, &  9, &  12, &  15, &  14 \} \\
        S_7 =&  \{13, &  11, &  4, &  1, &  3, &  15, &  5, &  9, &  0, &  10, &  14, &  7, &  6, &  8, &  2, &  12 \} \\
        S_8 =&  \{1, &  15, &  13, &  0, &  5, &  7, &  10, &  4, &  9, &  2, &  3, &  14, &  6, &  11, &  8, &  12 \}
    \end{array}
\end{equation}


\chapter{LIST OF PUBLICATIONS}

\begingroup
\renewcommand{\chapter}[2]{}%
\begin{thebibliography}{1}
\providecommand*{\BibEmph}[1]{#1}
\providecommand*{\cyrdash}{\hbox to.8em{--\hss--}}
\providecommand*{\BibDash}{\ifdim\lastskip>0pt\unskip\nobreak\hskip.2em\fi\cyrdash\hskip.2em\ignorespaces}

\bibitem{Kiyanchuk:DESSERT:2012}
\BibEmph{Oliynykov~R.~V., Kiyanchuk~R.~I.} {Perspective Symmetric Block Cipher
  optimized for Hardware Implementation}~// {6-th International Conference
  ``Dependable Systems, Services \& Technologies (DESSERT'12)''}. \BibDash
\newblock 2012.

\bibitem{Kiyanchuk:visnyk:2012}
\BibEmph{Kiyanchuk~R.~I., Oliynykov~R.~V.} {Linear transformation properties of
  ZUC cipher}~// \BibEmph{Visnyk}. \BibDash
\newblock 2012. \BibDash
\newblock {Mathematical modeling. Information technologies. Computer-aided
  control systems.}

\bibitem{karazina:zuc}
\BibEmph{{Kiyanchuk, R. I. and Oliynykov R. V.}} {Linear transformation
  properties of ZUC cipher}~// {Computer modeling in high-end technologies}~/
  Kharkiv national university of radio electronics. \BibDash
\newblock {Kharkiv}, 2012. \BibDash
\newblock P.~199 -- 202.

\bibitem{Kiyanchuk:2012:Banking}
\BibEmph{Kiyanchuk~R.~I.} {Differential analysis of S-functions
  [In~Russian]}~// {Scientific youth researching for European integration}~/
  Kharkiv university of banking. \BibDash
\newblock Kharkiv, 2012. \BibDash
\newblock Electronic resource on CD-ROM.

\bibitem{Kiyanchuk:2012:MMF}
\BibEmph{Kiyanchuk~R.~I.} {Differential analysis of S-functions
  [In~Russian]}~// {Radioelectronics and youth in XXI century}~/ Kharkiv
  national university of radio electronics. \BibDash
\newblock Kharkiv, 2012. \BibDash
\newblock {С.}~130 -- 131.

\bibitem{Kiyanchuk:2011:MMF}
\BibEmph{Kiyanchuk~R.~I.} {Comparative analysis of IDEA-like Block Symmetric
  Ciphers [In~Ukrainian]}~// {International Conference ``Computer
  Engineering''}~/ {Kharkiv National University of Radio Electronics}. \BibDash
\newblock Т.~5. \BibDash
\newblock Kharkiv, 2011. \BibDash April. \BibDash
\newblock {С.}~225 -- 227.

\bibitem{Kiyanchuk:IREF:2011:present}
\BibEmph{Oliynykov~R.~V., Kiyanchuk~R.~I.} {Perspective Symmetric Block Cipher
  Optimized for Hardware Implementation [In~Russian]}~// {``Telecommunication
  Systems and Technologies''}~/ {Kharkiv National University of Radio
  Electronics}. \BibDash
\newblock Vol.~II. \BibDash
\newblock {Kharkiv, Ukraine}, 2011. \BibDash October. \BibDash
\newblock P.~321 -- 330.

\bibitem{Kiyanchuk:2011:Customs}
\BibEmph{Oliynykov~R.~V., Kiyanchuk~R.~I.} {Usage of T-functions in Symmetric
  Cryptographic Transformations [In~Russian]}~// {``Perspectives of Information
  and transport-customs technologies in customs affairs, external economic
  industry and organizations management''}~/ {Kharkiv National University of
  Radio Electronics}. \BibDash
\newblock Dnipropetrovs'k, 2011. \BibDash December. \BibDash
\newblock {С.}~213 -- 215. \BibDash
\newblock Section 2.

\bibitem{Kiyanchuk:2009:rijndael}
\BibEmph{Dolgov~V.~I., Lysytska~I.~V., Kiyanchuk~R.~I.} {RIJNDAEL -- Is This a
  New or Well Forgotten Old Solution? [In~Russian]}~// {Computer Science and
  Technologies}~/ {Kharkiv National University of Radio Electronics}. \BibDash
\newblock 2009. \BibDash
\newblock P.~32 -- 35.

\end{thebibliography}
\endgroup

\end{appendices}

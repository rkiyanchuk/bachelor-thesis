
\chapter{LABOUR PROTECTION}
\label{sec:labour}

PC users working conditions are analyzed for
their compliance with normative documents on safety engineering  and
sanitation. For retrieving and evaluating the influence of possible dangerous
or harmful production factors an interaction system
``Human--Machine--Environment'' (HME) is developed. Safety measures are
developed as the result of such system analysis.

\section{Analysis of the workplace working conditions}

The administrative building in which the workspace is located situates on the
first floor and has dimensions \sloppy{$5 \times 5 \times 2.7 \, \text{m}$} and
its volume is~$67.5 \, \text{m}^3$ accordingly.

There are two workers in the office that use the following equipment:
\begin{itemize}
    \item Personal computer, 2 pieces;
    \item LCD monitor, 2 pieces.;
    \item Printer.
\end{itemize}
The analysis of the workspace indicates that the office has sufficient area
($6 \, \text{m}^2$) and volume ($20 \, \text{m}^2$) per worker and is compliant
with the sanity standard~\cite{dsanpin} for working with electronic computer displays.

The ``Human-Machine--Environment'' system (figure~\ref{fig:hme-graph}) is
created with the intention to retrieve possible dangerous and harmful
production factors~\cite{npaop4-12}. The list of H-M-E system connections is
shown in table~\ref{tbl:hme-legend}. Its main elements are:

\begin{description}
    \item[``Human''] --- staff that works in the office (2 workers);
    \item[``Machine''] --- electronic computers;
    \item[``Environment''] --- workspace environment;
    \item[``Product''] --- product of the algebraic cryptanalysis task solution.
\end{description}
Each element of the system contains three functional parts. Thereby the
``Human'' element is split by:
\begin{description}
    \item[$H_1$] --- execution of certain actions targeted on the main task solution;
    \item[$H_2$] --- human as a biological object influencing the environment;
    \item[$H_3$] --- human from the point of view of her psychophysiologic condition. 
\end{description}
The main goal of the ``Machine'' element is influencing the product ---
execution of the primary technological function. Its auxiliary functions are to
generate particular environment parameters and execute emergency surveillance.
\begin{description}
    \item[$M_1$] --- execution of the primary technological function
        (performing computations using SAGE algebra system, construction and
        analysis of algebraic quadratic system that defines the GOST~28147-89
        cipher);
    \item[$M_2$] --- emergency protection support;
    \item[$M_3$] --- influencing the human and the environment.
\end{description}

\begin{figure}[htbp]
	\centering
    %&latex
% Copyright 2011 Zoresvit (c) <zoresvit@gmail.com>
% 
% 
%/

\begin{tikzpicture}
    \tikzstyle{func} = [circle, draw=blue!40, fill=blue!20, thick, minimum size = 8mm]
    \tikzstyle{elem} = [circle, draw=green!40, fill=green!20, thick, minimum size = 8mm]
    \tikzstyle{bound} = [node distance=2em]
    \tikzstyle{backfill} = [fill=blue!10, rounded corners]
    \tikzstyle{linkbelow} = [bend right, looseness=0.7]
    \tikzstyle{linkabove} = [bend left, looseness=1.2]
    \tikzstyle{linklabel} = [circle, fill=cyan!10, minimum size=5mm, inner sep=0, draw, above, sloped, midway]

    \node (H1) [node distance=3ex] {\Large $H^1$};
    \node[func] (H11) [below=of H1]  {$H^1_1$};
    \node[func] (H12) [below=of H11] {$H^1_2$};
    \node[func] (H13) [below=of H12] {$H^1_3$};
    \node[bound] (H1bound) [left=of H1] {};

    \node (M1) [node distance=0.55\textwidth, right=of H1]  {\Large $M^1$};
    \node[func] (M11) [below=of M1]  {$M^1_1$};
    \node[func] (M12) [below=of M11] {$M^1_2$};
    \node[func] (M13) [below=of M12] {$M^1_3$};
    \node[bound] (M1bound) [right=of M1] {};

    \node (H2) [node distance=3ex, below=of H13] {\Large $H^2$};
    \node[func] (H21) [below=of H2]  {$H^2_1$};
    \node[func] (H22) [below=of H21] {$H^2_2$}; 
    \node[func] (H23) [below=of H22] {$H^2_3$}; 
    \node[bound] (H2bound) [left=of H2] {};

    \node (M2) [node distance=3ex, below=of M13]  {\Large $M^2$};
    \node[func] (M21) [below=of M2]  {$M^2_1$};
    \node[func] (M22) [below=of M21] {$M^2_2$};
    \node[func] (M23) [below=of M22] {$M^2_3$};
    \node[bound] (M2bound) [right=of M2] {};

    \node[elem] (Env) [node distance=0.18\textwidth, right=of H1] {Environment};
    \node[elem] (Prod) [node distance=0.2\textwidth, right=of H23] {Product};


    \foreach \H / \M in {H11/M11, H21/M21, H21/M11, H11/M21}{
    \draw (\H) to [->, linkbelow] (\M) node[linklabel]{1};
    }

    \foreach \M / \Prod in {M11/Prod, M21/Prod}{
    \draw (\M) to [->, linkbelow] (\Prod) node[linklabel]{2};
    }

    \foreach \H / \Prod in {H11/Prod, H21/Prod}{
    \draw (\H) to [<->, linkabove] (\Prod) node[linklabel]{3};
    }

    \foreach \H / \Env in {H12/Env, H22/Env}{
    \draw (\H) to [->, linkbelow] (\Env) node[linklabel]{4};
    }

    \foreach \Env / \H in {Env/H13, Env/H23}{
    \draw (\H) to [<-, linkbelow] (\Env) node[linklabel]{5};
    }

    \foreach \Prod / \H in {Prod/H13, Prod/H23}{
    \draw (\Prod) to [->, linkbelow] (\H) node[linklabel]{6};
    }

    \foreach \M / \Env in {M13/Env, M23/Env}{
    \draw (\M) to [->, linkabove] (\Env) node[linklabel]{7};
    }

    \foreach \Hact / \Hpsy in {H11/H13, H21/H23}{
    \draw (\Hact) to [->, linkabove] (\Hpsy) node[linklabel]{8};
    }

    \foreach \Hpsy / \Hbio in {H13/H12, H23/H22}{
    \draw (\Hpsy) to [->, linkabove] (\Hbio) node[linklabel]{9};
    }

    \foreach \H / \M in {H11/M13, H21/M23}{
    \draw (\H) to [->, linkabove] (\M) node[linklabel]{10};
    }

    \foreach \Env / \M in {Env/M12, Env/M22}{
    \draw (\Env) to [->, linkbelow] (\M) node[linklabel]{11};
    }

    \draw (H13) to [<->, linkabove] (H23) node[linklabel]{12};
    \draw (M13) to [<->, linkbelow] (M23) node[linklabel]{13};

    \begin{pgfonlayer}{background}
        \node [backfill, fit= (H1) (H11) (H12) (H13) (H1bound)] {};
        \node [backfill, fit= (M1) (M11) (M12) (M13) (M1bound)] {};
        \node [backfill, fit= (H2) (H21) (H22) (H23) (H2bound)] {};
        \node [backfill, fit= (M2) (M21) (M22) (M23) (M2bound)] {};
    \end{pgfonlayer}
\end{tikzpicture}


	\caption{H-M-E system scheme}
	\label{fig:hme-graph}
\end{figure}

\begin{longtable}{|p{0.09\textwidth}|p{0.16\textwidth}|p{0.66\textwidth}|}
\caption{Human--Machine--Enviroment system connections} \label{tbl:hme-legend} \\ \hline
\begin{center} Index \end{center} & Connection direction & \begin{center} Essence of the connections \end{center} \\ \hline
\endfirsthead
\multicolumn{3}{l}{\hspace*{5ex}Proceeding table \thechapter.\arabic{table}}
\endhead
    1 & $H_1 \rightarrow M_1$ & Human controls equipment providing its correct
    functioning (operating on computer using peripheral input devices).  \\ \hline
    2 & $M_1 \rightarrow \text{Prod}$ & Machine influence on the product
    (construction of the cryptographic transformation model, research
    computations). \\ \hline
    3 & $H_1 \leftrightarrow \text{Prod}$ & Human influence on the product and
    backwards (human is the source of ideas that defines needed actions during
    the work and analysis the working process; depending on the project
    complexity and accomplishment success causes mental strain for the
    workers). \\ \hline
    4 & $H_2 \rightarrow \text{Env}$ & Human influence on the environment
    (generation of heat and humidity, temperature increase). \\ \hline
    5 & $\text{Env} \rightarrow H_3$ & Environment influence on human health
    (the psychophysiologic condition working capacity depends on the
    environment conditions). \\ \hline
    6 & $\text{Prod} \rightarrow H_3$ & Product influence on human (complexity
    and execution progress of the primary task influence the human physical
    condition). \\ \hline
    7 & $M_3 \rightarrow \text{Env}$ & Machine influence on environment
    (computer is the source of increased temperature, noise, ionizing and
    electromagnetic emission). \\ \hline
    8 & $H_1 \rightarrow H_3$ & The working activity influences
    human psychophysiologic condition (human psychophysiologic condition
    depends on the success and amount of the completed work). \\ \hline

    9 & $H_3 \rightarrow H_2$ & Influence of psychophysiologic condition on
    human biological processes density (fatigue, irritability, euphoria). \\ \hline 
    10 & $H_1 \rightarrow M_3$ & 
    Human control of the machine influence on environment (normalization of
    harmful emission, humidity and temperature control). \\ \hline
    11 & $\text{Env} \rightarrow M_2$ & Environment influence on emergency
    protection abilities (unfavourable environment parameters may lead to
    insufficient emergency protection). \\ \hline
    12 & $H^*_3 \leftrightarrow H^*_3$ & Reciprocal influence of each human's
    psychophysiologic conditions. \\ \hline
    13 & $M^*_1 \leftrightarrow M^*_1$ & Mutual influence of machine side
    emissions. \\ \hline
\end{longtable}

Based on the analysed connections harmful and dangerous production factors may
be retrieved according to~\cite{gost003}. The following harmful factors are
found:
\begin{itemize}
    \item physical: 
        \begin{enumerate}
            \item increased electromagnetic emission;
            \item increased equipment surface temperature;
            \item reduced air humidity;
        \end{enumerate}
    \item chemical: absent;
    \item biological: absent;
    \item psychophysiologic: 
        \begin{enumerate}
            \item mental overstress;
            \item overstress of analyzers.
        \end{enumerate}
\end{itemize}
Reduced air humidity is the dominant harmful factor.

\section{Industrial security in production building}
Considering the electrical equipment located in the office, organizational and
technical countermeasures for protecting workers' safety need to be applied.
The workspace belongs to the class of increased fire threat according
to~\cite{npaop1-21}.

The building power supply is three-phased four-wired electric network with
dead-earthed neutral, alternating current, 50~Hz frequency and $380/220$~V
voltage. Neutral grounding of the equipment is done for providing electrical
security according to~\cite{gost030}. The current threshold for automatic
overload control exceeds the maximum equipment current consumption 5 times.
Triggering time of the overload controller does not exceed 0.2~s.

\section{Labour health at the workplace}

According to~\cite{dsanpin} the amount of harmful chemical pollutants in
computer rooms must not exceed maximum concentration of harmful pollutants in
free air. The level of positive and negative aerons must satisfy the
regulations in table~\ref{tbl:aero-req}.

\begin{table}[htbp]
    \centering
    \caption{Ionization levels}
    \label{tbl:aero-req}
    \begin{tabular}{|l|l|l|}
        \hline
        \multirow{2}{*}{Levels} & \multicolumn{2}{l|}{Number of ions in $1 \, \text{cm}^3$ of air} \\ \cline{2-3}
        & $n+$ & $n-$ \\ \hline
        Minimal & 400 & 600 \\ \hline
        Optimal & 1500 -- 3000 & 3000 -- 5000 \\ \hline
        Maximal & 50000 & 50000 \\ \hline
    \end{tabular}
\end{table}

An air exchange rate and ventilation duct parameters need to be estimated for
normalizing air quality. According to~\cite{snip} the air consumption level per
human amounts to at least $60 \, \text{m}^3/\text{h}$. For an office with two
workers the air consumption is $120 \, \text{m}^3/\text{h}$.

The volume of the air induction is computed according to \eqref{eqn:air}.
\begin{equation}
    \label{eqn:air}
    L = k \cdot V \enspace, 
\end{equation}
where $k$ --- recommended air factor ($k=1$ for this case), $V$ --- room
volume. Then the air consumption equals
$L = k \cdot V = k \cdot 5 \cdot 5 \cdot 2.7 = 67.5 \, \text{m}^3/\text{h}$.
The inducting air is not enough for efficient air exchange. The larger
induction volume $120 \, \text{m}^3/\text{h}$ is assumed. Taking natural
ventilation into account (air speed is less than $1 \, \text{m/s}$) the needed
size of the ventilation duct is \mbox{$140 \times 270$~mm}:
\begin{equation}
    \nu = 120 / 3600 / 0.14 / 0.27 = 0.88 \, \text{m/s} \enspace.
\end{equation}

\section{Fire safety of the production building}
Solid ignitable materials are used in the room, the building has brick walls
and armoured concrete floor. Therefore the building belongs to 2-nd fire
resistance ratio according to~\cite{dbn_b11}, production belongs to category B
of flammability and explosion risk according to~\cite{napb002}.

Pursuant to~\cite{napb001} the building must have two fire extinguishers with
up to 6~kg of reactant or one fire extinguisher with at least 8~kg of reactant.
від 8~кг. Any type of fire extinguishers fits for the category B building.

Evacuation plans that are located in every room and in corridors allow to
efficiently leave the building in case of fire through the main exit of 
 $1.9 \, \text{m}$ hight and $1.2 \, \text{m}$ wide. Two employees work at the
 office with area of $25 \, \text{m}^2$. Therefore no extra emergency exit is
 needed. The fire and explosion safety requirements are satisfied.

%be&latex
% Copyright 2011 Zoresvit (c) <zoresvit@gmail.com>

\documentclass[10pt, ucs]{beamer}
\usefonttheme[onlymath]{serif}
\usetheme{zslides}
\usepackage{xcolor}
\usepackage{tikz}

%\usepackage{pgfpages}
%\pgfpagesuselayout{2 on 1}[a4paper,border shrink=5mm]
%\mode<handout>{\setbeamercolor{background canvas}{bg=black!5}}

\overviewsection

\graphicspath{{images/}}
\setbeamercolor{alerted text}{fg=red!50!black}

\addtobeamertemplate{title page}{%
\vspace*{2ex}\Large\centering Kharkiv National University of Radio Electronics \\
}{}

\newcommand{\worktitle}{Cryptographic properties analysis \\ of symmetric ciphers \\[2ex]}
\newcommand{\workauthors}{Ruslan Kiyanchuk}


\title[Symmetric ciphers cryptographic properties]{\worktitle}
\author[\workauthors]{
    \texorpdfstring{Ruslan Kiyanchuk\\[2ex]
    \scriptsize\url{ruslan.kiyanchuk@gmail.com}}{\workauthors} \\[4ex]%
    \begin{flushleft}
    \hspace*{10em}
    \normalsize Supervisor: \hspace*{10ex} Halimov G. Z. \\[0.8ex]
    \hspace*{7em}
    \normalsize Consultant: \hspace*{10ex} Oliynykov R. V.
    \end{flushleft} \\[-8ex]
}

\subtitle{Bachelor Thesis}
\institute[KNURE]{}
\date[Kharkiv 2012]{\normalsize Kharkiv 2012}

\hypersetup{
pdftitle = {\worktitle},
pdfauthor = {\workauthors},
pdfkeywords = {}
}


\begin{document}
\selectlanguage{english}
\maketitle

\section{Importance}
\begin{frame}{Importance}{Current problems}
    \begin{block}{}
        \begin{itemize}
            \item Modern cryptoalgorithms use advanced mathematics.
            \item Real security systems often have weaknesses due to incorrect
                usage or implementation errors.
        \end{itemize}
    \end{block}
    \alert{GSM standard:}
    \begin{description}
        \item[A5/1] can be broken in seconds (using rainbow tables).
    \end{description}
    \alert{Satellite phones:}
    \begin{description}
        \item[GMR-1] based on A5/2 (prohibited for use since 2007).
        \item[GMR-2] proprietary algorithm (attack requires $65$~b keystream).
    \end{description}
    \alert{Wireless Internet:}
    \begin{description}
        \item [WEP] about $80000$ packets are needed for cryptanalysis.
        \item[E0] attack requires $2^{38}$ operations and $2^{23.8}$~data
            frames.
    \end{description}
    \begin{block}{}
        Deep analysis and public evaluation of ciphers is needed before deploying
        into real security systems.
    \end{block}
\end{frame}

\begin{frame}{Importance}{Cryptography evolution trends}
    \onslide<1->{
    \begin{block}{Lightweight cryptography}
        \begin{itemize}
            \item pervasive devices must provide data security;
            \item the computational resources are constrained;
            \item ciphers with efficient hardware implementation are needed.
        \end{itemize}
    \end{block}
    }
    \onslide<2->{
    \begin{block}{Mobile communication standards}
        \begin{itemize}
            \item new ciphers are needed for next generation LTE security;
            \item ZUC is Chinese stream cipher developed in 2011;
            \item the cipher is undergoing public evaluation period.
        \end{itemize}
    \end{block}
    }
    \onslide<3->{
    \begin{block}{GOST~28147 revisited}
        \begin{itemize}
            \item proposed for ISO standardization in 2010;
            \item is perspective for lightweight cryptography purposes;
            \item lacks strength evaluation against algebraic cryptanalysis.
        \end{itemize}
    \end{block}
    }
\end{frame}

\section{ZUC cryptographic properties}
\begin{frame}{ZUC stream cipher}{Perspective for usage in LTE}
    \begin{minipage}{0.4 \textwidth}
        \begin{itemize}
            \item 16 elements over {\small $GF(p) = GF(2^{31}-1)$};
            \item period $p^{16} -1 \approx 2^{496}$;
            \item $X_0 = s_{15_H} || s_{14_L}$,  \\
                $X_1 = s_{11_L} || s_{14_H}$,  \\
                $X_2 = s_{7_L} || s_{5_H}$,  \\
                $X_3 = s_{2_L} || s_{0_H}$;
            \item $R_1$, $R_2$ -- registers;
            \item $L_1$, $L_2$
                linear transformations --
                polynomials over $GF(2)[x]/(x^{32}+1)$;
            \item $S \rightarrow (S_0, S_1, S_0, S_1)$.
        \end{itemize}
        \begin{block}{}
            Negligible defect is revealed in linear transformations.
        \end{block}
    \end{minipage}%
        \hspace*{1.6ex}%
    \begin{minipage}{0.6 \textwidth}
        % Graphic for TeX using PGF
% Title: /home/zoresvit/Documents/research/diploma/graphics/zuc.dia
% Creator: Dia v0.97.1
% CreationDate: Fri Nov 11 01:43:40 2011
% For: zoresvit
% \usepackage{tikz}
% The following commands are not supported in PSTricks at present
% We define them conditionally, so when they are implemented,
% this pgf file will use them.
\ifx\du\undefined
  \newlength{\du}
\fi
\setlength{\du}{10\unitlength}

\begin{tikzpicture}[scale=0.55,every node/.style={scale=0.6}]

\pgftransformxscale{1.000000}
\pgftransformyscale{-1.000000}
\definecolor{dialinecolor}{rgb}{0.000000, 0.000000, 0.000000}
\pgfsetstrokecolor{dialinecolor}
\definecolor{dialinecolor}{rgb}{1.000000, 1.000000, 1.000000}
\pgfsetfillcolor{dialinecolor}
\pgfsetlinewidth{0.100000\du}
\pgfsetdash{{\pgflinewidth}{0.200000\du}}{0cm}
\pgfsetdash{{\pgflinewidth}{0.200000\du}}{0cm}
\pgfsetmiterjoin
\definecolor{dialinecolor}{rgb}{0.000000, 0.000000, 0.000000}
\pgfsetstrokecolor{dialinecolor}
\draw (7.000000\du,-3.000000\du)--(7.000000\du,7.000000\du)--(43.500000\du,7.000000\du)--(43.500000\du,-3.000000\du)--cycle;
% setfont left to latex
\definecolor{dialinecolor}{rgb}{0.000000, 0.000000, 0.000000}
\pgfsetstrokecolor{dialinecolor}
\node at (25.250000\du,2.195000\du){};
\pgfsetlinewidth{0.100000\du}
\pgfsetdash{{\pgflinewidth}{0.200000\du}}{0cm}
\pgfsetdash{{\pgflinewidth}{0.200000\du}}{0cm}
\pgfsetmiterjoin
\definecolor{dialinecolor}{rgb}{0.000000, 0.000000, 0.000000}
\pgfsetstrokecolor{dialinecolor}
\draw (7.000000\du,8.000000\du)--(7.000000\du,12.000000\du)--(43.500000\du,12.000000\du)--(43.500000\du,8.000000\du)--cycle;
% setfont left to latex
\definecolor{dialinecolor}{rgb}{0.000000, 0.000000, 0.000000}
\pgfsetstrokecolor{dialinecolor}
\node at (25.250000\du,10.195000\du){};
\pgfsetlinewidth{0.100000\du}
\pgfsetdash{{\pgflinewidth}{0.200000\du}}{0cm}
\pgfsetdash{{\pgflinewidth}{0.200000\du}}{0cm}
\pgfsetmiterjoin
\definecolor{dialinecolor}{rgb}{0.000000, 0.000000, 0.000000}
\pgfsetstrokecolor{dialinecolor}
\draw (7.100000\du,13.000000\du)--(7.100000\du,34.000000\du)--(32.100000\du,34.000000\du)--(32.100000\du,13.000000\du)--cycle;
% setfont left to latex
\definecolor{dialinecolor}{rgb}{0.000000, 0.000000, 0.000000}
\pgfsetstrokecolor{dialinecolor}
\node at (19.600000\du,23.695000\du){};
% setfont left to latex
\definecolor{dialinecolor}{rgb}{0.000000, 0.000000, 0.000000}
\pgfsetstrokecolor{dialinecolor}
\node[anchor=west] at (25.500000\du,1.500000\du){LFSR};
% setfont left to latex
\definecolor{dialinecolor}{rgb}{0.000000, 0.000000, 0.000000}
\pgfsetstrokecolor{dialinecolor}
\node[anchor=west] at (24.500000\du,9.000000\du){BR};
\definecolor{dialinecolor}{rgb}{1.000000, 1.000000, 1.000000}
\pgfsetfillcolor{dialinecolor}
\fill (40.000000\du,4.000000\du)--(40.000000\du,6.000000\du)--(42.000000\du,6.000000\du)--(42.000000\du,4.000000\du)--cycle;
\pgfsetlinewidth{0.100000\du}
\pgfsetdash{}{0pt}
\pgfsetdash{}{0pt}
\pgfsetmiterjoin
\definecolor{dialinecolor}{rgb}{0.000000, 0.000000, 0.000000}
\pgfsetstrokecolor{dialinecolor}
\draw (40.000000\du,4.000000\du)--(40.000000\du,6.000000\du)--(42.000000\du,6.000000\du)--(42.000000\du,4.000000\du)--cycle;
% setfont left to latex
\definecolor{dialinecolor}{rgb}{0.000000, 0.000000, 0.000000}
\pgfsetstrokecolor{dialinecolor}
\node at (41.000000\du,5.195000\du){$s_{0}$};
\definecolor{dialinecolor}{rgb}{1.000000, 1.000000, 1.000000}
\pgfsetfillcolor{dialinecolor}
\fill (38.000000\du,4.000000\du)--(38.000000\du,6.000000\du)--(40.000000\du,6.000000\du)--(40.000000\du,4.000000\du)--cycle;
\pgfsetlinewidth{0.100000\du}
\pgfsetdash{}{0pt}
\pgfsetdash{}{0pt}
\pgfsetmiterjoin
\definecolor{dialinecolor}{rgb}{0.000000, 0.000000, 0.000000}
\pgfsetstrokecolor{dialinecolor}
\draw (38.000000\du,4.000000\du)--(38.000000\du,6.000000\du)--(40.000000\du,6.000000\du)--(40.000000\du,4.000000\du)--cycle;
% setfont left to latex
\definecolor{dialinecolor}{rgb}{0.000000, 0.000000, 0.000000}
\pgfsetstrokecolor{dialinecolor}
\node at (39.000000\du,5.195000\du){$s_{1}$};
\definecolor{dialinecolor}{rgb}{1.000000, 1.000000, 1.000000}
\pgfsetfillcolor{dialinecolor}
\fill (36.000000\du,4.000000\du)--(36.000000\du,6.000000\du)--(38.000000\du,6.000000\du)--(38.000000\du,4.000000\du)--cycle;
\pgfsetlinewidth{0.100000\du}
\pgfsetdash{}{0pt}
\pgfsetdash{}{0pt}
\pgfsetmiterjoin
\definecolor{dialinecolor}{rgb}{0.000000, 0.000000, 0.000000}
\pgfsetstrokecolor{dialinecolor}
\draw (36.000000\du,4.000000\du)--(36.000000\du,6.000000\du)--(38.000000\du,6.000000\du)--(38.000000\du,4.000000\du)--cycle;
% setfont left to latex
\definecolor{dialinecolor}{rgb}{0.000000, 0.000000, 0.000000}
\pgfsetstrokecolor{dialinecolor}
\node at (37.000000\du,5.195000\du){$s_{2}$};
\definecolor{dialinecolor}{rgb}{1.000000, 1.000000, 1.000000}
\pgfsetfillcolor{dialinecolor}
\fill (34.000000\du,4.000000\du)--(34.000000\du,6.000000\du)--(36.000000\du,6.000000\du)--(36.000000\du,4.000000\du)--cycle;
\pgfsetlinewidth{0.100000\du}
\pgfsetdash{}{0pt}
\pgfsetdash{}{0pt}
\pgfsetmiterjoin
\definecolor{dialinecolor}{rgb}{0.000000, 0.000000, 0.000000}
\pgfsetstrokecolor{dialinecolor}
\draw (34.000000\du,4.000000\du)--(34.000000\du,6.000000\du)--(36.000000\du,6.000000\du)--(36.000000\du,4.000000\du)--cycle;
% setfont left to latex
\definecolor{dialinecolor}{rgb}{0.000000, 0.000000, 0.000000}
\pgfsetstrokecolor{dialinecolor}
\node at (35.000000\du,5.195000\du){$s_{3}$};
\definecolor{dialinecolor}{rgb}{1.000000, 1.000000, 1.000000}
\pgfsetfillcolor{dialinecolor}
\fill (32.000000\du,4.000000\du)--(32.000000\du,6.000000\du)--(34.000000\du,6.000000\du)--(34.000000\du,4.000000\du)--cycle;
\pgfsetlinewidth{0.100000\du}
\pgfsetdash{}{0pt}
\pgfsetdash{}{0pt}
\pgfsetmiterjoin
\definecolor{dialinecolor}{rgb}{0.000000, 0.000000, 0.000000}
\pgfsetstrokecolor{dialinecolor}
\draw (32.000000\du,4.000000\du)--(32.000000\du,6.000000\du)--(34.000000\du,6.000000\du)--(34.000000\du,4.000000\du)--cycle;
% setfont left to latex
\definecolor{dialinecolor}{rgb}{0.000000, 0.000000, 0.000000}
\pgfsetstrokecolor{dialinecolor}
\node at (33.000000\du,5.195000\du){$s_{4}$};
\definecolor{dialinecolor}{rgb}{1.000000, 1.000000, 1.000000}
\pgfsetfillcolor{dialinecolor}
\fill (30.000000\du,4.000000\du)--(30.000000\du,6.000000\du)--(32.000000\du,6.000000\du)--(32.000000\du,4.000000\du)--cycle;
\pgfsetlinewidth{0.100000\du}
\pgfsetdash{}{0pt}
\pgfsetdash{}{0pt}
\pgfsetmiterjoin
\definecolor{dialinecolor}{rgb}{0.000000, 0.000000, 0.000000}
\pgfsetstrokecolor{dialinecolor}
\draw (30.000000\du,4.000000\du)--(30.000000\du,6.000000\du)--(32.000000\du,6.000000\du)--(32.000000\du,4.000000\du)--cycle;
% setfont left to latex
\definecolor{dialinecolor}{rgb}{0.000000, 0.000000, 0.000000}
\pgfsetstrokecolor{dialinecolor}
\node at (31.000000\du,5.195000\du){$s_{5}$};
\definecolor{dialinecolor}{rgb}{1.000000, 1.000000, 1.000000}
\pgfsetfillcolor{dialinecolor}
\fill (28.000000\du,4.000000\du)--(28.000000\du,6.000000\du)--(30.000000\du,6.000000\du)--(30.000000\du,4.000000\du)--cycle;
\pgfsetlinewidth{0.100000\du}
\pgfsetdash{}{0pt}
\pgfsetdash{}{0pt}
\pgfsetmiterjoin
\definecolor{dialinecolor}{rgb}{0.000000, 0.000000, 0.000000}
\pgfsetstrokecolor{dialinecolor}
\draw (28.000000\du,4.000000\du)--(28.000000\du,6.000000\du)--(30.000000\du,6.000000\du)--(30.000000\du,4.000000\du)--cycle;
% setfont left to latex
\definecolor{dialinecolor}{rgb}{0.000000, 0.000000, 0.000000}
\pgfsetstrokecolor{dialinecolor}
\node at (29.000000\du,5.195000\du){$s_{6}$};
\definecolor{dialinecolor}{rgb}{1.000000, 1.000000, 1.000000}
\pgfsetfillcolor{dialinecolor}
\fill (26.000000\du,4.000000\du)--(26.000000\du,6.000000\du)--(28.000000\du,6.000000\du)--(28.000000\du,4.000000\du)--cycle;
\pgfsetlinewidth{0.100000\du}
\pgfsetdash{}{0pt}
\pgfsetdash{}{0pt}
\pgfsetmiterjoin
\definecolor{dialinecolor}{rgb}{0.000000, 0.000000, 0.000000}
\pgfsetstrokecolor{dialinecolor}
\draw (26.000000\du,4.000000\du)--(26.000000\du,6.000000\du)--(28.000000\du,6.000000\du)--(28.000000\du,4.000000\du)--cycle;
% setfont left to latex
\definecolor{dialinecolor}{rgb}{0.000000, 0.000000, 0.000000}
\pgfsetstrokecolor{dialinecolor}
\node at (27.000000\du,5.195000\du){$s_{7}$};
\definecolor{dialinecolor}{rgb}{1.000000, 1.000000, 1.000000}
\pgfsetfillcolor{dialinecolor}
\fill (24.000000\du,4.000000\du)--(24.000000\du,6.000000\du)--(26.000000\du,6.000000\du)--(26.000000\du,4.000000\du)--cycle;
\pgfsetlinewidth{0.100000\du}
\pgfsetdash{}{0pt}
\pgfsetdash{}{0pt}
\pgfsetmiterjoin
\definecolor{dialinecolor}{rgb}{0.000000, 0.000000, 0.000000}
\pgfsetstrokecolor{dialinecolor}
\draw (24.000000\du,4.000000\du)--(24.000000\du,6.000000\du)--(26.000000\du,6.000000\du)--(26.000000\du,4.000000\du)--cycle;
% setfont left to latex
\definecolor{dialinecolor}{rgb}{0.000000, 0.000000, 0.000000}
\pgfsetstrokecolor{dialinecolor}
\node at (25.000000\du,5.195000\du){$s_{8}$};
\definecolor{dialinecolor}{rgb}{1.000000, 1.000000, 1.000000}
\pgfsetfillcolor{dialinecolor}
\fill (22.000000\du,4.000000\du)--(22.000000\du,6.000000\du)--(24.000000\du,6.000000\du)--(24.000000\du,4.000000\du)--cycle;
\pgfsetlinewidth{0.100000\du}
\pgfsetdash{}{0pt}
\pgfsetdash{}{0pt}
\pgfsetmiterjoin
\definecolor{dialinecolor}{rgb}{0.000000, 0.000000, 0.000000}
\pgfsetstrokecolor{dialinecolor}
\draw (22.000000\du,4.000000\du)--(22.000000\du,6.000000\du)--(24.000000\du,6.000000\du)--(24.000000\du,4.000000\du)--cycle;
% setfont left to latex
\definecolor{dialinecolor}{rgb}{0.000000, 0.000000, 0.000000}
\pgfsetstrokecolor{dialinecolor}
\node at (23.000000\du,5.195000\du){$s_{9}$};
\definecolor{dialinecolor}{rgb}{1.000000, 1.000000, 1.000000}
\pgfsetfillcolor{dialinecolor}
\fill (19.800000\du,4.000000\du)--(19.800000\du,6.000000\du)--(22.050000\du,6.000000\du)--(22.050000\du,4.000000\du)--cycle;
\pgfsetlinewidth{0.100000\du}
\pgfsetdash{}{0pt}
\pgfsetdash{}{0pt}
\pgfsetmiterjoin
\definecolor{dialinecolor}{rgb}{0.000000, 0.000000, 0.000000}
\pgfsetstrokecolor{dialinecolor}
\draw (19.800000\du,4.000000\du)--(19.800000\du,6.000000\du)--(22.050000\du,6.000000\du)--(22.050000\du,4.000000\du)--cycle;
% setfont left to latex
\definecolor{dialinecolor}{rgb}{0.000000, 0.000000, 0.000000}
\pgfsetstrokecolor{dialinecolor}
\node at (20.925000\du,5.195000\du){$s_{10}$};
\definecolor{dialinecolor}{rgb}{1.000000, 1.000000, 1.000000}
\pgfsetfillcolor{dialinecolor}
\fill (17.800000\du,4.000000\du)--(17.800000\du,6.000000\du)--(20.050000\du,6.000000\du)--(20.050000\du,4.000000\du)--cycle;
\pgfsetlinewidth{0.100000\du}
\pgfsetdash{}{0pt}
\pgfsetdash{}{0pt}
\pgfsetmiterjoin
\definecolor{dialinecolor}{rgb}{0.000000, 0.000000, 0.000000}
\pgfsetstrokecolor{dialinecolor}
\draw (17.800000\du,4.000000\du)--(17.800000\du,6.000000\du)--(20.050000\du,6.000000\du)--(20.050000\du,4.000000\du)--cycle;
% setfont left to latex
\definecolor{dialinecolor}{rgb}{0.000000, 0.000000, 0.000000}
\pgfsetstrokecolor{dialinecolor}
\node at (18.925000\du,5.195000\du){$s_{11}$};
\definecolor{dialinecolor}{rgb}{1.000000, 1.000000, 1.000000}
\pgfsetfillcolor{dialinecolor}
\fill (15.600000\du,4.000000\du)--(15.600000\du,6.000000\du)--(17.850000\du,6.000000\du)--(17.850000\du,4.000000\du)--cycle;
\pgfsetlinewidth{0.100000\du}
\pgfsetdash{}{0pt}
\pgfsetdash{}{0pt}
\pgfsetmiterjoin
\definecolor{dialinecolor}{rgb}{0.000000, 0.000000, 0.000000}
\pgfsetstrokecolor{dialinecolor}
\draw (15.600000\du,4.000000\du)--(15.600000\du,6.000000\du)--(17.850000\du,6.000000\du)--(17.850000\du,4.000000\du)--cycle;
% setfont left to latex
\definecolor{dialinecolor}{rgb}{0.000000, 0.000000, 0.000000}
\pgfsetstrokecolor{dialinecolor}
\node at (16.725000\du,5.195000\du){$s_{12}$};
\definecolor{dialinecolor}{rgb}{1.000000, 1.000000, 1.000000}
\pgfsetfillcolor{dialinecolor}
\fill (13.475000\du,4.000000\du)--(13.475000\du,6.000000\du)--(15.725000\du,6.000000\du)--(15.725000\du,4.000000\du)--cycle;
\pgfsetlinewidth{0.100000\du}
\pgfsetdash{}{0pt}
\pgfsetdash{}{0pt}
\pgfsetmiterjoin
\definecolor{dialinecolor}{rgb}{0.000000, 0.000000, 0.000000}
\pgfsetstrokecolor{dialinecolor}
\draw (13.475000\du,4.000000\du)--(13.475000\du,6.000000\du)--(15.725000\du,6.000000\du)--(15.725000\du,4.000000\du)--cycle;
% setfont left to latex
\definecolor{dialinecolor}{rgb}{0.000000, 0.000000, 0.000000}
\pgfsetstrokecolor{dialinecolor}
\node at (14.600000\du,5.195000\du){$s_{13}$};
\definecolor{dialinecolor}{rgb}{1.000000, 1.000000, 1.000000}
\pgfsetfillcolor{dialinecolor}
\fill (11.275000\du,4.000000\du)--(11.275000\du,6.000000\du)--(13.525000\du,6.000000\du)--(13.525000\du,4.000000\du)--cycle;
\pgfsetlinewidth{0.100000\du}
\pgfsetdash{}{0pt}
\pgfsetdash{}{0pt}
\pgfsetmiterjoin
\definecolor{dialinecolor}{rgb}{0.000000, 0.000000, 0.000000}
\pgfsetstrokecolor{dialinecolor}
\draw (11.275000\du,4.000000\du)--(11.275000\du,6.000000\du)--(13.525000\du,6.000000\du)--(13.525000\du,4.000000\du)--cycle;
% setfont left to latex
\definecolor{dialinecolor}{rgb}{0.000000, 0.000000, 0.000000}
\pgfsetstrokecolor{dialinecolor}
\node at (12.400000\du,5.195000\du){$s_{14}$};
\definecolor{dialinecolor}{rgb}{1.000000, 1.000000, 1.000000}
\pgfsetfillcolor{dialinecolor}
\fill (9.000000\du,4.000000\du)--(9.000000\du,6.000000\du)--(11.250000\du,6.000000\du)--(11.250000\du,4.000000\du)--cycle;
\pgfsetlinewidth{0.100000\du}
\pgfsetdash{}{0pt}
\pgfsetdash{}{0pt}
\pgfsetmiterjoin
\definecolor{dialinecolor}{rgb}{0.000000, 0.000000, 0.000000}
\pgfsetstrokecolor{dialinecolor}
\draw (9.000000\du,4.000000\du)--(9.000000\du,6.000000\du)--(11.250000\du,6.000000\du)--(11.250000\du,4.000000\du)--cycle;
% setfont left to latex
\definecolor{dialinecolor}{rgb}{0.000000, 0.000000, 0.000000}
\pgfsetstrokecolor{dialinecolor}
\node at (10.125000\du,5.195000\du){$s_{15}$};
\definecolor{dialinecolor}{rgb}{1.000000, 1.000000, 1.000000}
\pgfsetfillcolor{dialinecolor}
\fill (9.000000\du,-2.000000\du)--(9.000000\du,0.000000\du)--(42.000000\du,0.000000\du)--(42.000000\du,-2.000000\du)--cycle;
\pgfsetlinewidth{0.100000\du}
\pgfsetdash{}{0pt}
\pgfsetdash{}{0pt}
\pgfsetmiterjoin
\definecolor{dialinecolor}{rgb}{0.000000, 0.000000, 0.000000}
\pgfsetstrokecolor{dialinecolor}
\draw (9.000000\du,-2.000000\du)--(9.000000\du,0.000000\du)--(42.000000\du,0.000000\du)--(42.000000\du,-2.000000\du)--cycle;
% setfont left to latex
\definecolor{dialinecolor}{rgb}{0.000000, 0.000000, 0.000000}
\pgfsetstrokecolor{dialinecolor}
\node at (25.500000\du,-0.805000\du){$\mod 2^{31} - 1$};
\pgfsetlinewidth{0.100000\du}
\pgfsetdash{}{0pt}
\pgfsetdash{}{0pt}
\pgfsetbuttcap
\pgfsetmiterjoin
\pgfsetlinewidth{0.100000\du}
\pgfsetbuttcap
\pgfsetmiterjoin
\pgfsetdash{}{0pt}
\definecolor{dialinecolor}{rgb}{1.000000, 1.000000, 1.000000}
\pgfsetfillcolor{dialinecolor}
\pgfpathmoveto{\pgfpoint{39.655000\du}{1.200000\du}}
\pgfpathlineto{\pgfpoint{41.475000\du}{1.200000\du}}
\pgfpathcurveto{\pgfpoint{41.726290\du}{1.200000\du}}{\pgfpoint{41.930000\du}{1.513401\du}}{\pgfpoint{41.930000\du}{1.900000\du}}
\pgfpathcurveto{\pgfpoint{41.930000\du}{2.286600\du}}{\pgfpoint{41.726290\du}{2.600000\du}}{\pgfpoint{41.475000\du}{2.600000\du}}
\pgfpathlineto{\pgfpoint{39.655000\du}{2.600000\du}}
\pgfpathcurveto{\pgfpoint{39.403710\du}{2.600000\du}}{\pgfpoint{39.200000\du}{2.286600\du}}{\pgfpoint{39.200000\du}{1.900000\du}}
\pgfpathcurveto{\pgfpoint{39.200000\du}{1.513401\du}}{\pgfpoint{39.403710\du}{1.200000\du}}{\pgfpoint{39.655000\du}{1.200000\du}}
\pgfusepath{fill}
\definecolor{dialinecolor}{rgb}{0.000000, 0.000000, 0.000000}
\pgfsetstrokecolor{dialinecolor}
\pgfpathmoveto{\pgfpoint{39.655000\du}{1.200000\du}}
\pgfpathlineto{\pgfpoint{41.475000\du}{1.200000\du}}
\pgfpathcurveto{\pgfpoint{41.726290\du}{1.200000\du}}{\pgfpoint{41.930000\du}{1.513401\du}}{\pgfpoint{41.930000\du}{1.900000\du}}
\pgfpathcurveto{\pgfpoint{41.930000\du}{2.286600\du}}{\pgfpoint{41.726290\du}{2.600000\du}}{\pgfpoint{41.475000\du}{2.600000\du}}
\pgfpathlineto{\pgfpoint{39.655000\du}{2.600000\du}}
\pgfpathcurveto{\pgfpoint{39.403710\du}{2.600000\du}}{\pgfpoint{39.200000\du}{2.286600\du}}{\pgfpoint{39.200000\du}{1.900000\du}}
\pgfpathcurveto{\pgfpoint{39.200000\du}{1.513401\du}}{\pgfpoint{39.403710\du}{1.200000\du}}{\pgfpoint{39.655000\du}{1.200000\du}}
\pgfusepath{stroke}
% setfont left to latex
\definecolor{dialinecolor}{rgb}{0.000000, 0.000000, 0.000000}
\pgfsetstrokecolor{dialinecolor}
\node at (40.565000\du,2.032292\du){$\scriptstyle 2^8 + 1$};
\pgfsetlinewidth{0.100000\du}
\pgfsetdash{}{0pt}
\pgfsetdash{}{0pt}
\pgfsetbuttcap
{
\definecolor{dialinecolor}{rgb}{0.000000, 0.000000, 0.000000}
\pgfsetfillcolor{dialinecolor}
% was here!!!
\pgfsetarrowsend{latex}
\definecolor{dialinecolor}{rgb}{0.000000, 0.000000, 0.000000}
\pgfsetstrokecolor{dialinecolor}
\draw (41.000000\du,3.949707\du)--(41.000000\du,2.600000\du);
}
\pgfsetlinewidth{0.100000\du}
\pgfsetdash{}{0pt}
\pgfsetdash{}{0pt}
\pgfsetbuttcap
{
\definecolor{dialinecolor}{rgb}{0.000000, 0.000000, 0.000000}
\pgfsetfillcolor{dialinecolor}
% was here!!!
\pgfsetarrowsend{latex}
\definecolor{dialinecolor}{rgb}{0.000000, 0.000000, 0.000000}
\pgfsetstrokecolor{dialinecolor}
\draw (41.020000\du,1.200000\du)--(41.000000\du,0.000000\du);
}
\pgfsetlinewidth{0.100000\du}
\pgfsetdash{}{0pt}
\pgfsetdash{}{0pt}
\pgfsetbuttcap
\pgfsetmiterjoin
\pgfsetlinewidth{0.100000\du}
\pgfsetbuttcap
\pgfsetmiterjoin
\pgfsetdash{}{0pt}
\definecolor{dialinecolor}{rgb}{1.000000, 1.000000, 1.000000}
\pgfsetfillcolor{dialinecolor}
\pgfpathmoveto{\pgfpoint{32.366250\du}{1.200000\du}}
\pgfpathlineto{\pgfpoint{33.831250\du}{1.200000\du}}
\pgfpathcurveto{\pgfpoint{34.033524\du}{1.200000\du}}{\pgfpoint{34.197500\du}{1.513401\du}}{\pgfpoint{34.197500\du}{1.900000\du}}
\pgfpathcurveto{\pgfpoint{34.197500\du}{2.286600\du}}{\pgfpoint{34.033524\du}{2.600000\du}}{\pgfpoint{33.831250\du}{2.600000\du}}
\pgfpathlineto{\pgfpoint{32.366250\du}{2.600000\du}}
\pgfpathcurveto{\pgfpoint{32.163976\du}{2.600000\du}}{\pgfpoint{32.000000\du}{2.286600\du}}{\pgfpoint{32.000000\du}{1.900000\du}}
\pgfpathcurveto{\pgfpoint{32.000000\du}{1.513401\du}}{\pgfpoint{32.163976\du}{1.200000\du}}{\pgfpoint{32.366250\du}{1.200000\du}}
\pgfusepath{fill}
\definecolor{dialinecolor}{rgb}{0.000000, 0.000000, 0.000000}
\pgfsetstrokecolor{dialinecolor}
\pgfpathmoveto{\pgfpoint{32.366250\du}{1.200000\du}}
\pgfpathlineto{\pgfpoint{33.831250\du}{1.200000\du}}
\pgfpathcurveto{\pgfpoint{34.033524\du}{1.200000\du}}{\pgfpoint{34.197500\du}{1.513401\du}}{\pgfpoint{34.197500\du}{1.900000\du}}
\pgfpathcurveto{\pgfpoint{34.197500\du}{2.286600\du}}{\pgfpoint{34.033524\du}{2.600000\du}}{\pgfpoint{33.831250\du}{2.600000\du}}
\pgfpathlineto{\pgfpoint{32.366250\du}{2.600000\du}}
\pgfpathcurveto{\pgfpoint{32.163976\du}{2.600000\du}}{\pgfpoint{32.000000\du}{2.286600\du}}{\pgfpoint{32.000000\du}{1.900000\du}}
\pgfpathcurveto{\pgfpoint{32.000000\du}{1.513401\du}}{\pgfpoint{32.163976\du}{1.200000\du}}{\pgfpoint{32.366250\du}{1.200000\du}}
\pgfusepath{stroke}
% setfont left to latex
\definecolor{dialinecolor}{rgb}{0.000000, 0.000000, 0.000000}
\pgfsetstrokecolor{dialinecolor}
\node at (33.098750\du,2.032292\du){$2^{20}$};
\pgfsetlinewidth{0.100000\du}
\pgfsetdash{}{0pt}
\pgfsetdash{}{0pt}
\pgfsetbuttcap
{
\definecolor{dialinecolor}{rgb}{0.000000, 0.000000, 0.000000}
\pgfsetfillcolor{dialinecolor}
% was here!!!
\pgfsetarrowsend{latex}
\definecolor{dialinecolor}{rgb}{0.000000, 0.000000, 0.000000}
\pgfsetstrokecolor{dialinecolor}
\draw (33.000000\du,4.000000\du)--(33.000000\du,2.600000\du);
}
\pgfsetlinewidth{0.100000\du}
\pgfsetdash{}{0pt}
\pgfsetdash{}{0pt}
\pgfsetbuttcap
{
\definecolor{dialinecolor}{rgb}{0.000000, 0.000000, 0.000000}
\pgfsetfillcolor{dialinecolor}
% was here!!!
\pgfsetarrowsend{latex}
\definecolor{dialinecolor}{rgb}{0.000000, 0.000000, 0.000000}
\pgfsetstrokecolor{dialinecolor}
\draw (33.000000\du,1.200000\du)--(33.000000\du,0.000000\du);
}
\pgfsetlinewidth{0.100000\du}
\pgfsetdash{}{0pt}
\pgfsetdash{}{0pt}
\pgfsetbuttcap
\pgfsetmiterjoin
\pgfsetlinewidth{0.100000\du}
\pgfsetbuttcap
\pgfsetmiterjoin
\pgfsetdash{}{0pt}
\definecolor{dialinecolor}{rgb}{1.000000, 1.000000, 1.000000}
\pgfsetfillcolor{dialinecolor}
\pgfpathmoveto{\pgfpoint{20.366250\du}{1.200000\du}}
\pgfpathlineto{\pgfpoint{21.831250\du}{1.200000\du}}
\pgfpathcurveto{\pgfpoint{22.033524\du}{1.200000\du}}{\pgfpoint{22.197500\du}{1.513401\du}}{\pgfpoint{22.197500\du}{1.900000\du}}
\pgfpathcurveto{\pgfpoint{22.197500\du}{2.286600\du}}{\pgfpoint{22.033524\du}{2.600000\du}}{\pgfpoint{21.831250\du}{2.600000\du}}
\pgfpathlineto{\pgfpoint{20.366250\du}{2.600000\du}}
\pgfpathcurveto{\pgfpoint{20.163976\du}{2.600000\du}}{\pgfpoint{20.000000\du}{2.286600\du}}{\pgfpoint{20.000000\du}{1.900000\du}}
\pgfpathcurveto{\pgfpoint{20.000000\du}{1.513401\du}}{\pgfpoint{20.163976\du}{1.200000\du}}{\pgfpoint{20.366250\du}{1.200000\du}}
\pgfusepath{fill}
\definecolor{dialinecolor}{rgb}{0.000000, 0.000000, 0.000000}
\pgfsetstrokecolor{dialinecolor}
\pgfpathmoveto{\pgfpoint{20.366250\du}{1.200000\du}}
\pgfpathlineto{\pgfpoint{21.831250\du}{1.200000\du}}
\pgfpathcurveto{\pgfpoint{22.033524\du}{1.200000\du}}{\pgfpoint{22.197500\du}{1.513401\du}}{\pgfpoint{22.197500\du}{1.900000\du}}
\pgfpathcurveto{\pgfpoint{22.197500\du}{2.286600\du}}{\pgfpoint{22.033524\du}{2.600000\du}}{\pgfpoint{21.831250\du}{2.600000\du}}
\pgfpathlineto{\pgfpoint{20.366250\du}{2.600000\du}}
\pgfpathcurveto{\pgfpoint{20.163976\du}{2.600000\du}}{\pgfpoint{20.000000\du}{2.286600\du}}{\pgfpoint{20.000000\du}{1.900000\du}}
\pgfpathcurveto{\pgfpoint{20.000000\du}{1.513401\du}}{\pgfpoint{20.163976\du}{1.200000\du}}{\pgfpoint{20.366250\du}{1.200000\du}}
\pgfusepath{stroke}
% setfont left to latex
\definecolor{dialinecolor}{rgb}{0.000000, 0.000000, 0.000000}
\pgfsetstrokecolor{dialinecolor}
\node at (21.098750\du,2.032292\du){$2^{21}$};
\pgfsetlinewidth{0.100000\du}
\pgfsetdash{}{0pt}
\pgfsetdash{}{0pt}
\pgfsetbuttcap
{
\definecolor{dialinecolor}{rgb}{0.000000, 0.000000, 0.000000}
\pgfsetfillcolor{dialinecolor}
% was here!!!
\pgfsetarrowsend{latex}
\definecolor{dialinecolor}{rgb}{0.000000, 0.000000, 0.000000}
\pgfsetstrokecolor{dialinecolor}
\draw (21.200000\du,4.000000\du)--(21.200000\du,2.600000\du);
}
\pgfsetlinewidth{0.100000\du}
\pgfsetdash{}{0pt}
\pgfsetdash{}{0pt}
\pgfsetbuttcap
{
\definecolor{dialinecolor}{rgb}{0.000000, 0.000000, 0.000000}
\pgfsetfillcolor{dialinecolor}
% was here!!!
\pgfsetarrowsend{latex}
\definecolor{dialinecolor}{rgb}{0.000000, 0.000000, 0.000000}
\pgfsetstrokecolor{dialinecolor}
\draw (21.200000\du,1.200000\du)--(21.200000\du,0.000000\du);
}
\pgfsetlinewidth{0.100000\du}
\pgfsetdash{}{0pt}
\pgfsetdash{}{0pt}
\pgfsetbuttcap
\pgfsetmiterjoin
\pgfsetlinewidth{0.100000\du}
\pgfsetbuttcap
\pgfsetmiterjoin
\pgfsetdash{}{0pt}
\definecolor{dialinecolor}{rgb}{1.000000, 1.000000, 1.000000}
\pgfsetfillcolor{dialinecolor}
\pgfpathmoveto{\pgfpoint{13.966250\du}{1.200000\du}}
\pgfpathlineto{\pgfpoint{15.431250\du}{1.200000\du}}
\pgfpathcurveto{\pgfpoint{15.633524\du}{1.200000\du}}{\pgfpoint{15.797500\du}{1.513401\du}}{\pgfpoint{15.797500\du}{1.900000\du}}
\pgfpathcurveto{\pgfpoint{15.797500\du}{2.286600\du}}{\pgfpoint{15.633524\du}{2.600000\du}}{\pgfpoint{15.431250\du}{2.600000\du}}
\pgfpathlineto{\pgfpoint{13.966250\du}{2.600000\du}}
\pgfpathcurveto{\pgfpoint{13.763976\du}{2.600000\du}}{\pgfpoint{13.600000\du}{2.286600\du}}{\pgfpoint{13.600000\du}{1.900000\du}}
\pgfpathcurveto{\pgfpoint{13.600000\du}{1.513401\du}}{\pgfpoint{13.763976\du}{1.200000\du}}{\pgfpoint{13.966250\du}{1.200000\du}}
\pgfusepath{fill}
\definecolor{dialinecolor}{rgb}{0.000000, 0.000000, 0.000000}
\pgfsetstrokecolor{dialinecolor}
\pgfpathmoveto{\pgfpoint{13.966250\du}{1.200000\du}}
\pgfpathlineto{\pgfpoint{15.431250\du}{1.200000\du}}
\pgfpathcurveto{\pgfpoint{15.633524\du}{1.200000\du}}{\pgfpoint{15.797500\du}{1.513401\du}}{\pgfpoint{15.797500\du}{1.900000\du}}
\pgfpathcurveto{\pgfpoint{15.797500\du}{2.286600\du}}{\pgfpoint{15.633524\du}{2.600000\du}}{\pgfpoint{15.431250\du}{2.600000\du}}
\pgfpathlineto{\pgfpoint{13.966250\du}{2.600000\du}}
\pgfpathcurveto{\pgfpoint{13.763976\du}{2.600000\du}}{\pgfpoint{13.600000\du}{2.286600\du}}{\pgfpoint{13.600000\du}{1.900000\du}}
\pgfpathcurveto{\pgfpoint{13.600000\du}{1.513401\du}}{\pgfpoint{13.763976\du}{1.200000\du}}{\pgfpoint{13.966250\du}{1.200000\du}}
\pgfusepath{stroke}
% setfont left to latex
\definecolor{dialinecolor}{rgb}{0.000000, 0.000000, 0.000000}
\pgfsetstrokecolor{dialinecolor}
\node at (14.698750\du,2.032292\du){$2^{15}$};
\pgfsetlinewidth{0.100000\du}
\pgfsetdash{}{0pt}
\pgfsetdash{}{0pt}
\pgfsetbuttcap
{
\definecolor{dialinecolor}{rgb}{0.000000, 0.000000, 0.000000}
\pgfsetfillcolor{dialinecolor}
% was here!!!
\pgfsetarrowsend{latex}
\definecolor{dialinecolor}{rgb}{0.000000, 0.000000, 0.000000}
\pgfsetstrokecolor{dialinecolor}
\draw (14.600000\du,4.000000\du)--(14.600000\du,2.600000\du);
}
\pgfsetlinewidth{0.100000\du}
\pgfsetdash{}{0pt}
\pgfsetdash{}{0pt}
\pgfsetbuttcap
{
\definecolor{dialinecolor}{rgb}{0.000000, 0.000000, 0.000000}
\pgfsetfillcolor{dialinecolor}
% was here!!!
\pgfsetarrowsend{latex}
\definecolor{dialinecolor}{rgb}{0.000000, 0.000000, 0.000000}
\pgfsetstrokecolor{dialinecolor}
\draw (14.600000\du,1.200000\du)--(14.600000\du,0.000000\du);
}
\pgfsetlinewidth{0.100000\du}
\pgfsetdash{}{0pt}
\pgfsetdash{}{0pt}
\pgfsetbuttcap
\pgfsetmiterjoin
\pgfsetlinewidth{0.100000\du}
\pgfsetbuttcap
\pgfsetmiterjoin
\pgfsetdash{}{0pt}
\definecolor{dialinecolor}{rgb}{1.000000, 1.000000, 1.000000}
\pgfsetfillcolor{dialinecolor}
\pgfpathmoveto{\pgfpoint{9.366250\du}{1.200000\du}}
\pgfpathlineto{\pgfpoint{10.831250\du}{1.200000\du}}
\pgfpathcurveto{\pgfpoint{11.033524\du}{1.200000\du}}{\pgfpoint{11.197500\du}{1.513401\du}}{\pgfpoint{11.197500\du}{1.900000\du}}
\pgfpathcurveto{\pgfpoint{11.197500\du}{2.286600\du}}{\pgfpoint{11.033524\du}{2.600000\du}}{\pgfpoint{10.831250\du}{2.600000\du}}
\pgfpathlineto{\pgfpoint{9.366250\du}{2.600000\du}}
\pgfpathcurveto{\pgfpoint{9.163976\du}{2.600000\du}}{\pgfpoint{9.000000\du}{2.286600\du}}{\pgfpoint{9.000000\du}{1.900000\du}}
\pgfpathcurveto{\pgfpoint{9.000000\du}{1.513401\du}}{\pgfpoint{9.163976\du}{1.200000\du}}{\pgfpoint{9.366250\du}{1.200000\du}}
\pgfusepath{fill}
\definecolor{dialinecolor}{rgb}{0.000000, 0.000000, 0.000000}
\pgfsetstrokecolor{dialinecolor}
\pgfpathmoveto{\pgfpoint{9.366250\du}{1.200000\du}}
\pgfpathlineto{\pgfpoint{10.831250\du}{1.200000\du}}
\pgfpathcurveto{\pgfpoint{11.033524\du}{1.200000\du}}{\pgfpoint{11.197500\du}{1.513401\du}}{\pgfpoint{11.197500\du}{1.900000\du}}
\pgfpathcurveto{\pgfpoint{11.197500\du}{2.286600\du}}{\pgfpoint{11.033524\du}{2.600000\du}}{\pgfpoint{10.831250\du}{2.600000\du}}
\pgfpathlineto{\pgfpoint{9.366250\du}{2.600000\du}}
\pgfpathcurveto{\pgfpoint{9.163976\du}{2.600000\du}}{\pgfpoint{9.000000\du}{2.286600\du}}{\pgfpoint{9.000000\du}{1.900000\du}}
\pgfpathcurveto{\pgfpoint{9.000000\du}{1.513401\du}}{\pgfpoint{9.163976\du}{1.200000\du}}{\pgfpoint{9.366250\du}{1.200000\du}}
\pgfusepath{stroke}
% setfont left to latex
\definecolor{dialinecolor}{rgb}{0.000000, 0.000000, 0.000000}
\pgfsetstrokecolor{dialinecolor}
\node at (10.098750\du,2.032292\du){$2^{17}$};
\pgfsetlinewidth{0.100000\du}
\pgfsetdash{}{0pt}
\pgfsetdash{}{0pt}
\pgfsetbuttcap
{
\definecolor{dialinecolor}{rgb}{0.000000, 0.000000, 0.000000}
\pgfsetfillcolor{dialinecolor}
% was here!!!
\pgfsetarrowsend{latex}
\definecolor{dialinecolor}{rgb}{0.000000, 0.000000, 0.000000}
\pgfsetstrokecolor{dialinecolor}
\draw (10.200000\du,4.000000\du)--(10.200000\du,2.600000\du);
}
\pgfsetlinewidth{0.100000\du}
\pgfsetdash{}{0pt}
\pgfsetdash{}{0pt}
\pgfsetbuttcap
{
\definecolor{dialinecolor}{rgb}{0.000000, 0.000000, 0.000000}
\pgfsetfillcolor{dialinecolor}
% was here!!!
\pgfsetarrowsend{latex}
\definecolor{dialinecolor}{rgb}{0.000000, 0.000000, 0.000000}
\pgfsetstrokecolor{dialinecolor}
\draw (10.200000\du,1.200000\du)--(10.200000\du,0.000000\du);
}
\pgfsetlinewidth{0.100000\du}
\pgfsetdash{}{0pt}
\pgfsetdash{}{0pt}
\pgfsetmiterjoin
\pgfsetbuttcap
{
\definecolor{dialinecolor}{rgb}{0.000000, 0.000000, 0.000000}
\pgfsetfillcolor{dialinecolor}
% was here!!!
\pgfsetarrowsend{latex}
{\pgfsetcornersarced{\pgfpoint{0.000000\du}{0.000000\du}}\definecolor{dialinecolor}{rgb}{0.000000, 0.000000, 0.000000}
\pgfsetstrokecolor{dialinecolor}
\draw (9.000000\du,-1.000000\du)--(7.500000\du,-1.000000\du)--(7.500000\du,5.000000\du)--(9.000000\du,5.000000\du);
}}
\definecolor{dialinecolor}{rgb}{1.000000, 1.000000, 1.000000}
\pgfsetfillcolor{dialinecolor}
\fill (37.000000\du,9.000000\du)--(37.000000\du,11.000000\du)--(41.000000\du,11.000000\du)--(41.000000\du,9.000000\du)--cycle;
\pgfsetlinewidth{0.100000\du}
\pgfsetdash{}{0pt}
\pgfsetdash{}{0pt}
\pgfsetmiterjoin
\definecolor{dialinecolor}{rgb}{0.000000, 0.000000, 0.000000}
\pgfsetstrokecolor{dialinecolor}
\draw (37.000000\du,9.000000\du)--(37.000000\du,11.000000\du)--(41.000000\du,11.000000\du)--(41.000000\du,9.000000\du)--cycle;
% setfont left to latex
\definecolor{dialinecolor}{rgb}{0.000000, 0.000000, 0.000000}
\pgfsetstrokecolor{dialinecolor}
\node at (39.000000\du,9.595000\du){$16 \vdots 16$};
\definecolor{dialinecolor}{rgb}{1.000000, 1.000000, 1.000000}
\pgfsetfillcolor{dialinecolor}
\fill (27.000000\du,9.000000\du)--(27.000000\du,11.000000\du)--(31.000000\du,11.000000\du)--(31.000000\du,9.000000\du)--cycle;
\pgfsetlinewidth{0.100000\du}
\pgfsetdash{}{0pt}
\pgfsetdash{}{0pt}
\pgfsetmiterjoin
\definecolor{dialinecolor}{rgb}{0.000000, 0.000000, 0.000000}
\pgfsetstrokecolor{dialinecolor}
\draw (27.000000\du,9.000000\du)--(27.000000\du,11.000000\du)--(31.000000\du,11.000000\du)--(31.000000\du,9.000000\du)--cycle;
% setfont left to latex
\definecolor{dialinecolor}{rgb}{0.000000, 0.000000, 0.000000}
\pgfsetstrokecolor{dialinecolor}
\node at (29.000000\du,9.595000\du){$16 \vdots 16$};
\definecolor{dialinecolor}{rgb}{1.000000, 1.000000, 1.000000}
\pgfsetfillcolor{dialinecolor}
\fill (19.000000\du,9.000000\du)--(19.000000\du,11.000000\du)--(23.000000\du,11.000000\du)--(23.000000\du,9.000000\du)--cycle;
\pgfsetlinewidth{0.100000\du}
\pgfsetdash{}{0pt}
\pgfsetdash{}{0pt}
\pgfsetmiterjoin
\definecolor{dialinecolor}{rgb}{0.000000, 0.000000, 0.000000}
\pgfsetstrokecolor{dialinecolor}
\draw (19.000000\du,9.000000\du)--(19.000000\du,11.000000\du)--(23.000000\du,11.000000\du)--(23.000000\du,9.000000\du)--cycle;
% setfont left to latex
\definecolor{dialinecolor}{rgb}{0.000000, 0.000000, 0.000000}
\pgfsetstrokecolor{dialinecolor}
\node at (21.000000\du,9.595000\du){$16 \vdots 16$};
\definecolor{dialinecolor}{rgb}{1.000000, 1.000000, 1.000000}
\pgfsetfillcolor{dialinecolor}
\fill (9.000000\du,9.000000\du)--(9.000000\du,11.000000\du)--(13.000000\du,11.000000\du)--(13.000000\du,9.000000\du)--cycle;
\pgfsetlinewidth{0.100000\du}
\pgfsetdash{}{0pt}
\pgfsetdash{}{0pt}
\pgfsetmiterjoin
\definecolor{dialinecolor}{rgb}{0.000000, 0.000000, 0.000000}
\pgfsetstrokecolor{dialinecolor}
\draw (9.000000\du,9.000000\du)--(9.000000\du,11.000000\du)--(13.000000\du,11.000000\du)--(13.000000\du,9.000000\du)--cycle;
% setfont left to latex
\definecolor{dialinecolor}{rgb}{0.000000, 0.000000, 0.000000}
\pgfsetstrokecolor{dialinecolor}
\node at (11.000000\du,9.595000\du){$16 \vdots 16$};
% setfont left to latex
\definecolor{dialinecolor}{rgb}{0.000000, 0.000000, 0.000000}
\pgfsetstrokecolor{dialinecolor}
\node[anchor=west] at (13.500000\du,11.000000\du){$X_0$};
% setfont left to latex
\definecolor{dialinecolor}{rgb}{0.000000, 0.000000, 0.000000}
\pgfsetstrokecolor{dialinecolor}
\node[anchor=west] at (23.500000\du,11.000000\du){$X_1$};
% setfont left to latex
\definecolor{dialinecolor}{rgb}{0.000000, 0.000000, 0.000000}
\pgfsetstrokecolor{dialinecolor}
\node[anchor=west] at (31.500000\du,11.000000\du){$X_2$};
% setfont left to latex
\definecolor{dialinecolor}{rgb}{0.000000, 0.000000, 0.000000}
\pgfsetstrokecolor{dialinecolor}
\node[anchor=west] at (41.500000\du,11.000000\du){$X_3$};
\pgfsetlinewidth{0.100000\du}
\pgfsetdash{}{0pt}
\pgfsetdash{}{0pt}
\pgfsetbuttcap
{
\definecolor{dialinecolor}{rgb}{0.000000, 0.000000, 0.000000}
\pgfsetfillcolor{dialinecolor}
% was here!!!
\pgfsetarrowsend{latex}
\definecolor{dialinecolor}{rgb}{0.000000, 0.000000, 0.000000}
\pgfsetstrokecolor{dialinecolor}
\draw (41.000000\du,6.000000\du)--(40.000000\du,9.000000\du);
}
\pgfsetlinewidth{0.100000\du}
\pgfsetdash{}{0pt}
\pgfsetdash{}{0pt}
\pgfsetbuttcap
{
\definecolor{dialinecolor}{rgb}{0.000000, 0.000000, 0.000000}
\pgfsetfillcolor{dialinecolor}
% was here!!!
\pgfsetarrowsend{latex}
\definecolor{dialinecolor}{rgb}{0.000000, 0.000000, 0.000000}
\pgfsetstrokecolor{dialinecolor}
\draw (37.000000\du,6.000000\du)--(38.475586\du,8.951172\du);
}
\pgfsetlinewidth{0.100000\du}
\pgfsetdash{}{0pt}
\pgfsetdash{}{0pt}
\pgfsetbuttcap
{
\definecolor{dialinecolor}{rgb}{0.000000, 0.000000, 0.000000}
\pgfsetfillcolor{dialinecolor}
% was here!!!
\pgfsetarrowsend{latex}
\definecolor{dialinecolor}{rgb}{0.000000, 0.000000, 0.000000}
\pgfsetstrokecolor{dialinecolor}
\draw (31.000000\du,6.000000\du)--(30.000000\du,9.000000\du);
}
\pgfsetlinewidth{0.100000\du}
\pgfsetdash{}{0pt}
\pgfsetdash{}{0pt}
\pgfsetbuttcap
{
\definecolor{dialinecolor}{rgb}{0.000000, 0.000000, 0.000000}
\pgfsetfillcolor{dialinecolor}
% was here!!!
\pgfsetarrowsend{latex}
\definecolor{dialinecolor}{rgb}{0.000000, 0.000000, 0.000000}
\pgfsetstrokecolor{dialinecolor}
\draw (27.000000\du,6.000000\du)--(28.000000\du,9.000000\du);
}
\pgfsetlinewidth{0.100000\du}
\pgfsetdash{}{0pt}
\pgfsetdash{}{0pt}
\pgfsetbuttcap
{
\definecolor{dialinecolor}{rgb}{0.000000, 0.000000, 0.000000}
\pgfsetfillcolor{dialinecolor}
% was here!!!
\pgfsetarrowsend{latex}
\definecolor{dialinecolor}{rgb}{0.000000, 0.000000, 0.000000}
\pgfsetstrokecolor{dialinecolor}
\draw (23.000000\du,6.000000\du)--(22.000000\du,9.000000\du);
}
\pgfsetlinewidth{0.100000\du}
\pgfsetdash{}{0pt}
\pgfsetdash{}{0pt}
\pgfsetbuttcap
{
\definecolor{dialinecolor}{rgb}{0.000000, 0.000000, 0.000000}
\pgfsetfillcolor{dialinecolor}
% was here!!!
\pgfsetarrowsend{latex}
\definecolor{dialinecolor}{rgb}{0.000000, 0.000000, 0.000000}
\pgfsetstrokecolor{dialinecolor}
\draw (18.925000\du,6.000000\du)--(20.000000\du,9.000000\du);
}
\pgfsetlinewidth{0.100000\du}
\pgfsetdash{}{0pt}
\pgfsetdash{}{0pt}
\pgfsetbuttcap
{
\definecolor{dialinecolor}{rgb}{0.000000, 0.000000, 0.000000}
\pgfsetfillcolor{dialinecolor}
% was here!!!
\pgfsetarrowsend{latex}
\definecolor{dialinecolor}{rgb}{0.000000, 0.000000, 0.000000}
\pgfsetstrokecolor{dialinecolor}
\draw (10.125000\du,6.000000\du)--(10.000000\du,9.000000\du);
}
\pgfsetlinewidth{0.100000\du}
\pgfsetdash{}{0pt}
\pgfsetdash{}{0pt}
\pgfsetbuttcap
{
\definecolor{dialinecolor}{rgb}{0.000000, 0.000000, 0.000000}
\pgfsetfillcolor{dialinecolor}
% was here!!!
\pgfsetarrowsend{latex}
\definecolor{dialinecolor}{rgb}{0.000000, 0.000000, 0.000000}
\pgfsetstrokecolor{dialinecolor}
\draw (12.400000\du,6.000000\du)--(12.000000\du,9.000000\du);
}
\pgfsetlinewidth{0.100000\du}
\pgfsetdash{}{0pt}
\pgfsetdash{}{0pt}
\pgfsetbuttcap
\pgfsetmiterjoin
\pgfsetlinewidth{0.100000\du}
\pgfsetbuttcap
\pgfsetmiterjoin
\pgfsetdash{}{0pt}
\definecolor{dialinecolor}{rgb}{1.000000, 1.000000, 1.000000}
\pgfsetfillcolor{dialinecolor}
\fill (23.000000\du,14.000000\du)--(23.000000\du,15.500000\du)--(24.500000\du,15.500000\du)--(24.500000\du,14.000000\du)--cycle;
\definecolor{dialinecolor}{rgb}{0.000000, 0.000000, 0.000000}
\pgfsetstrokecolor{dialinecolor}
\draw (23.000000\du,14.000000\du)--(23.000000\du,15.500000\du)--(24.500000\du,15.500000\du)--(24.500000\du,14.000000\du)--cycle;
\pgfsetbuttcap
\pgfsetmiterjoin
\pgfsetdash{}{0pt}
\definecolor{dialinecolor}{rgb}{0.000000, 0.000000, 0.000000}
\pgfsetstrokecolor{dialinecolor}
\draw (23.750000\du,14.225000\du)--(23.750000\du,15.275000\du);
\pgfsetbuttcap
\pgfsetmiterjoin
\pgfsetdash{}{0pt}
\definecolor{dialinecolor}{rgb}{0.000000, 0.000000, 0.000000}
\pgfsetstrokecolor{dialinecolor}
\draw (23.225000\du,14.750000\du)--(24.275000\du,14.750000\du);
\pgfsetlinewidth{0.100000\du}
\pgfsetdash{}{0pt}
\pgfsetdash{}{0pt}
\pgfsetbuttcap
\pgfsetmiterjoin
\pgfsetlinewidth{0.100000\du}
\pgfsetbuttcap
\pgfsetmiterjoin
\pgfsetdash{}{0pt}
\definecolor{dialinecolor}{rgb}{1.000000, 1.000000, 1.000000}
\pgfsetfillcolor{dialinecolor}
\pgfpathellipse{\pgfpoint{10.990000\du}{14.750000\du}}{\pgfpoint{0.750000\du}{0\du}}{\pgfpoint{0\du}{0.750000\du}}
\pgfusepath{fill}
\definecolor{dialinecolor}{rgb}{0.000000, 0.000000, 0.000000}
\pgfsetstrokecolor{dialinecolor}
\pgfpathellipse{\pgfpoint{10.990000\du}{14.750000\du}}{\pgfpoint{0.750000\du}{0\du}}{\pgfpoint{0\du}{0.750000\du}}
\pgfusepath{stroke}
\pgfsetbuttcap
\pgfsetmiterjoin
\pgfsetdash{}{0pt}
\definecolor{dialinecolor}{rgb}{0.000000, 0.000000, 0.000000}
\pgfsetstrokecolor{dialinecolor}
\draw (10.990000\du,14.000000\du)--(10.990000\du,15.500000\du);
\pgfsetbuttcap
\pgfsetmiterjoin
\pgfsetdash{}{0pt}
\definecolor{dialinecolor}{rgb}{0.000000, 0.000000, 0.000000}
\pgfsetstrokecolor{dialinecolor}
\draw (10.240000\du,14.750000\du)--(11.740000\du,14.750000\du);
\pgfsetlinewidth{0.100000\du}
\pgfsetdash{}{0pt}
\pgfsetdash{}{0pt}
\pgfsetbuttcap
{
\definecolor{dialinecolor}{rgb}{0.000000, 0.000000, 0.000000}
\pgfsetfillcolor{dialinecolor}
% was here!!!
\pgfsetarrowsend{latex}
\definecolor{dialinecolor}{rgb}{0.000000, 0.000000, 0.000000}
\pgfsetstrokecolor{dialinecolor}
\draw (11.000000\du,11.000000\du)--(10.990000\du,14.000000\du);
}
\definecolor{dialinecolor}{rgb}{1.000000, 1.000000, 1.000000}
\pgfsetfillcolor{dialinecolor}
\fill (9.500000\du,17.000000\du)--(9.500000\du,19.000000\du)--(12.500000\du,19.000000\du)--(12.500000\du,17.000000\du)--cycle;
\pgfsetlinewidth{0.100000\du}
\pgfsetdash{}{0pt}
\pgfsetdash{}{0pt}
\pgfsetmiterjoin
\definecolor{dialinecolor}{rgb}{0.000000, 0.000000, 0.000000}
\pgfsetstrokecolor{dialinecolor}
\draw (9.500000\du,17.000000\du)--(9.500000\du,19.000000\du)--(12.500000\du,19.000000\du)--(12.500000\du,17.000000\du)--cycle;
% setfont left to latex
\definecolor{dialinecolor}{rgb}{0.000000, 0.000000, 0.000000}
\pgfsetstrokecolor{dialinecolor}
\node at (11.000000\du,18.195000\du){$R_1$};
\pgfsetlinewidth{0.100000\du}
\pgfsetdash{}{0pt}
\pgfsetdash{}{0pt}
\pgfsetbuttcap
{
\definecolor{dialinecolor}{rgb}{0.000000, 0.000000, 0.000000}
\pgfsetfillcolor{dialinecolor}
% was here!!!
\pgfsetarrowsend{latex}
\definecolor{dialinecolor}{rgb}{0.000000, 0.000000, 0.000000}
\pgfsetstrokecolor{dialinecolor}
\draw (11.000000\du,17.000000\du)--(10.990000\du,15.500000\du);
}
\pgfsetlinewidth{0.100000\du}
\pgfsetdash{}{0pt}
\pgfsetdash{}{0pt}
\pgfsetbuttcap
\pgfsetmiterjoin
\pgfsetlinewidth{0.100000\du}
\pgfsetbuttcap
\pgfsetmiterjoin
\pgfsetdash{}{0pt}
\definecolor{dialinecolor}{rgb}{1.000000, 1.000000, 1.000000}
\pgfsetfillcolor{dialinecolor}
\fill (10.200000\du,21.200000\du)--(10.200000\du,22.700000\du)--(11.700000\du,22.700000\du)--(11.700000\du,21.200000\du)--cycle;
\definecolor{dialinecolor}{rgb}{0.000000, 0.000000, 0.000000}
\pgfsetstrokecolor{dialinecolor}
\draw (10.200000\du,21.200000\du)--(10.200000\du,22.700000\du)--(11.700000\du,22.700000\du)--(11.700000\du,21.200000\du)--cycle;
\pgfsetbuttcap
\pgfsetmiterjoin
\pgfsetdash{}{0pt}
\definecolor{dialinecolor}{rgb}{0.000000, 0.000000, 0.000000}
\pgfsetstrokecolor{dialinecolor}
\draw (10.950000\du,21.425000\du)--(10.950000\du,22.475000\du);
\pgfsetbuttcap
\pgfsetmiterjoin
\pgfsetdash{}{0pt}
\definecolor{dialinecolor}{rgb}{0.000000, 0.000000, 0.000000}
\pgfsetstrokecolor{dialinecolor}
\draw (10.425000\du,21.950000\du)--(11.475000\du,21.950000\du);
\pgfsetlinewidth{0.100000\du}
\pgfsetdash{}{0pt}
\pgfsetdash{}{0pt}
\pgfsetbuttcap
{
\definecolor{dialinecolor}{rgb}{0.000000, 0.000000, 0.000000}
\pgfsetfillcolor{dialinecolor}
% was here!!!
\pgfsetarrowsend{latex}
\definecolor{dialinecolor}{rgb}{0.000000, 0.000000, 0.000000}
\pgfsetstrokecolor{dialinecolor}
\draw (11.000000\du,19.000000\du)--(10.950000\du,21.200000\du);
}
\pgfsetlinewidth{0.100000\du}
\pgfsetdash{}{0pt}
\pgfsetdash{}{0pt}
\pgfsetmiterjoin
\pgfsetbuttcap
{
\definecolor{dialinecolor}{rgb}{0.000000, 0.000000, 0.000000}
\pgfsetfillcolor{dialinecolor}
% was here!!!
\pgfsetarrowsend{latex}
{\pgfsetcornersarced{\pgfpoint{0.000000\du}{0.000000\du}}\definecolor{dialinecolor}{rgb}{0.000000, 0.000000, 0.000000}
\pgfsetstrokecolor{dialinecolor}
\draw (21.000000\du,11.000000\du)--(21.000000\du,11.000000\du)--(21.000000\du,18.000000\du)--(12.500000\du,18.000000\du);
}}
\pgfsetlinewidth{0.100000\du}
\pgfsetdash{}{0pt}
\pgfsetdash{}{0pt}
\pgfsetbuttcap
{
\definecolor{dialinecolor}{rgb}{0.000000, 0.000000, 0.000000}
\pgfsetfillcolor{dialinecolor}
% was here!!!
\definecolor{dialinecolor}{rgb}{0.000000, 0.000000, 0.000000}
\pgfsetstrokecolor{dialinecolor}
\pgfpathmoveto{\pgfpoint{21.400000\du}{14.800000\du}}
\pgfpatharc{360}{181}{0.400000\du and 0.400000\du}
\pgfusepath{stroke}
}
\pgfsetlinewidth{0.100000\du}
\pgfsetdash{}{0pt}
\pgfsetdash{}{0pt}
\pgfsetbuttcap
{
\definecolor{dialinecolor}{rgb}{0.000000, 0.000000, 0.000000}
\pgfsetfillcolor{dialinecolor}
% was here!!!
\definecolor{dialinecolor}{rgb}{0.000000, 0.000000, 0.000000}
\pgfsetstrokecolor{dialinecolor}
\draw (11.740000\du,14.750000\du)--(20.600000\du,14.800000\du);
}
\pgfsetlinewidth{0.100000\du}
\pgfsetdash{}{0pt}
\pgfsetdash{}{0pt}
\pgfsetbuttcap
{
\definecolor{dialinecolor}{rgb}{0.000000, 0.000000, 0.000000}
\pgfsetfillcolor{dialinecolor}
% was here!!!
\pgfsetarrowsend{latex}
\definecolor{dialinecolor}{rgb}{0.000000, 0.000000, 0.000000}
\pgfsetstrokecolor{dialinecolor}
\draw (21.400000\du,14.800000\du)--(23.000000\du,14.750000\du);
}
\definecolor{dialinecolor}{rgb}{1.000000, 1.000000, 1.000000}
\pgfsetfillcolor{dialinecolor}
\fill (22.400000\du,17.000000\du)--(22.400000\du,19.000000\du)--(25.400000\du,19.000000\du)--(25.400000\du,17.000000\du)--cycle;
\pgfsetlinewidth{0.100000\du}
\pgfsetdash{}{0pt}
\pgfsetdash{}{0pt}
\pgfsetmiterjoin
\definecolor{dialinecolor}{rgb}{0.000000, 0.000000, 0.000000}
\pgfsetstrokecolor{dialinecolor}
\draw (22.400000\du,17.000000\du)--(22.400000\du,19.000000\du)--(25.400000\du,19.000000\du)--(25.400000\du,17.000000\du)--cycle;
% setfont left to latex
\definecolor{dialinecolor}{rgb}{0.000000, 0.000000, 0.000000}
\pgfsetstrokecolor{dialinecolor}
\node at (23.900000\du,18.195000\du){$R_2$};
\pgfsetlinewidth{0.100000\du}
\pgfsetdash{}{0pt}
\pgfsetdash{}{0pt}
\pgfsetbuttcap
{
\definecolor{dialinecolor}{rgb}{0.000000, 0.000000, 0.000000}
\pgfsetfillcolor{dialinecolor}
% was here!!!
\pgfsetarrowsend{latex}
\definecolor{dialinecolor}{rgb}{0.000000, 0.000000, 0.000000}
\pgfsetstrokecolor{dialinecolor}
\draw (23.750000\du,16.980000\du)--(23.750000\du,15.500000\du);
}
\pgfsetlinewidth{0.100000\du}
\pgfsetdash{}{0pt}
\pgfsetdash{}{0pt}
\pgfsetbuttcap
\pgfsetmiterjoin
\pgfsetlinewidth{0.100000\du}
\pgfsetbuttcap
\pgfsetmiterjoin
\pgfsetdash{}{0pt}
\definecolor{dialinecolor}{rgb}{1.000000, 1.000000, 1.000000}
\pgfsetfillcolor{dialinecolor}
\pgfpathellipse{\pgfpoint{23.800000\du}{21.950000\du}}{\pgfpoint{0.750000\du}{0\du}}{\pgfpoint{0\du}{0.750000\du}}
\pgfusepath{fill}
\definecolor{dialinecolor}{rgb}{0.000000, 0.000000, 0.000000}
\pgfsetstrokecolor{dialinecolor}
\pgfpathellipse{\pgfpoint{23.800000\du}{21.950000\du}}{\pgfpoint{0.750000\du}{0\du}}{\pgfpoint{0\du}{0.750000\du}}
\pgfusepath{stroke}
\pgfsetbuttcap
\pgfsetmiterjoin
\pgfsetdash{}{0pt}
\definecolor{dialinecolor}{rgb}{0.000000, 0.000000, 0.000000}
\pgfsetstrokecolor{dialinecolor}
\draw (23.800000\du,21.200000\du)--(23.800000\du,22.700000\du);
\pgfsetbuttcap
\pgfsetmiterjoin
\pgfsetdash{}{0pt}
\definecolor{dialinecolor}{rgb}{0.000000, 0.000000, 0.000000}
\pgfsetstrokecolor{dialinecolor}
\draw (23.050000\du,21.950000\du)--(24.550000\du,21.950000\du);
\pgfsetlinewidth{0.100000\du}
\pgfsetdash{}{0pt}
\pgfsetdash{}{0pt}
\pgfsetbuttcap
{
\definecolor{dialinecolor}{rgb}{0.000000, 0.000000, 0.000000}
\pgfsetfillcolor{dialinecolor}
% was here!!!
\pgfsetarrowsend{latex}
\definecolor{dialinecolor}{rgb}{0.000000, 0.000000, 0.000000}
\pgfsetstrokecolor{dialinecolor}
\draw (23.900000\du,19.000000\du)--(23.800000\du,21.200000\du);
}
\definecolor{dialinecolor}{rgb}{1.000000, 1.000000, 1.000000}
\pgfsetfillcolor{dialinecolor}
\fill (9.400000\du,25.000000\du)--(9.400000\du,26.900000\du)--(25.400000\du,26.900000\du)--(25.400000\du,25.000000\du)--cycle;
\pgfsetlinewidth{0.100000\du}
\pgfsetdash{}{0pt}
\pgfsetdash{}{0pt}
\pgfsetmiterjoin
\definecolor{dialinecolor}{rgb}{0.000000, 0.000000, 0.000000}
\pgfsetstrokecolor{dialinecolor}
\draw (9.400000\du,25.000000\du)--(9.400000\du,26.900000\du)--(25.400000\du,26.900000\du)--(25.400000\du,25.000000\du)--cycle;
% setfont left to latex
\definecolor{dialinecolor}{rgb}{0.000000, 0.000000, 0.000000}
\pgfsetstrokecolor{dialinecolor}
\node at (17.400000\du,26.145000\du){$<<< 16$};
\pgfsetlinewidth{0.100000\du}
\pgfsetdash{}{0pt}
\pgfsetdash{}{0pt}
\pgfsetbuttcap
{
\definecolor{dialinecolor}{rgb}{0.000000, 0.000000, 0.000000}
\pgfsetfillcolor{dialinecolor}
% was here!!!
\pgfsetarrowsend{latex}
\definecolor{dialinecolor}{rgb}{0.000000, 0.000000, 0.000000}
\pgfsetstrokecolor{dialinecolor}
\draw (23.800000\du,22.700000\du)--(23.800000\du,25.000000\du);
}
\pgfsetlinewidth{0.100000\du}
\pgfsetdash{}{0pt}
\pgfsetdash{}{0pt}
\pgfsetbuttcap
{
\definecolor{dialinecolor}{rgb}{0.000000, 0.000000, 0.000000}
\pgfsetfillcolor{dialinecolor}
% was here!!!
\pgfsetarrowsend{latex}
\definecolor{dialinecolor}{rgb}{0.000000, 0.000000, 0.000000}
\pgfsetstrokecolor{dialinecolor}
\draw (10.950000\du,22.700000\du)--(10.958000\du,24.990125\du);
}
\definecolor{dialinecolor}{rgb}{1.000000, 1.000000, 1.000000}
\pgfsetfillcolor{dialinecolor}
\fill (9.400000\du,29.000000\du)--(9.400000\du,31.000000\du)--(12.400000\du,31.000000\du)--(12.400000\du,29.000000\du)--cycle;
\pgfsetlinewidth{0.100000\du}
\pgfsetdash{}{0pt}
\pgfsetdash{}{0pt}
\pgfsetmiterjoin
\definecolor{dialinecolor}{rgb}{0.000000, 0.000000, 0.000000}
\pgfsetstrokecolor{dialinecolor}
\draw (9.400000\du,29.000000\du)--(9.400000\du,31.000000\du)--(12.400000\du,31.000000\du)--(12.400000\du,29.000000\du)--cycle;
% setfont left to latex
\definecolor{dialinecolor}{rgb}{0.000000, 0.000000, 0.000000}
\pgfsetstrokecolor{dialinecolor}
\node at (10.900000\du,30.195000\du){$S \cdot L_1$};
\definecolor{dialinecolor}{rgb}{1.000000, 1.000000, 1.000000}
\pgfsetfillcolor{dialinecolor}
\fill (22.400000\du,29.000000\du)--(22.400000\du,31.000000\du)--(25.400000\du,31.000000\du)--(25.400000\du,29.000000\du)--cycle;
\pgfsetlinewidth{0.100000\du}
\pgfsetdash{}{0pt}
\pgfsetdash{}{0pt}
\pgfsetmiterjoin
\definecolor{dialinecolor}{rgb}{0.000000, 0.000000, 0.000000}
\pgfsetstrokecolor{dialinecolor}
\draw (22.400000\du,29.000000\du)--(22.400000\du,31.000000\du)--(25.400000\du,31.000000\du)--(25.400000\du,29.000000\du)--cycle;
% setfont left to latex
\definecolor{dialinecolor}{rgb}{0.000000, 0.000000, 0.000000}
\pgfsetstrokecolor{dialinecolor}
\node at (23.900000\du,30.195000\du){$S \cdot L_2$};
\pgfsetlinewidth{0.100000\du}
\pgfsetdash{}{0pt}
\pgfsetdash{}{0pt}
\pgfsetbuttcap
{
\definecolor{dialinecolor}{rgb}{0.000000, 0.000000, 0.000000}
\pgfsetfillcolor{dialinecolor}
% was here!!!
\pgfsetarrowsend{latex}
\definecolor{dialinecolor}{rgb}{0.000000, 0.000000, 0.000000}
\pgfsetstrokecolor{dialinecolor}
\draw (23.900000\du,26.900000\du)--(23.900000\du,29.000000\du);
}
\pgfsetlinewidth{0.100000\du}
\pgfsetdash{}{0pt}
\pgfsetdash{}{0pt}
\pgfsetbuttcap
{
\definecolor{dialinecolor}{rgb}{0.000000, 0.000000, 0.000000}
\pgfsetfillcolor{dialinecolor}
% was here!!!
\pgfsetarrowsend{latex}
\definecolor{dialinecolor}{rgb}{0.000000, 0.000000, 0.000000}
\pgfsetstrokecolor{dialinecolor}
\draw (10.900000\du,26.900000\du)--(10.900000\du,29.000000\du);
}
\pgfsetlinewidth{0.100000\du}
\pgfsetdash{}{0pt}
\pgfsetdash{}{0pt}
\pgfsetmiterjoin
\pgfsetbuttcap
{
\definecolor{dialinecolor}{rgb}{0.000000, 0.000000, 0.000000}
\pgfsetfillcolor{dialinecolor}
% was here!!!
\pgfsetarrowsend{latex}
{\pgfsetcornersarced{\pgfpoint{0.000000\du}{0.000000\du}}\definecolor{dialinecolor}{rgb}{0.000000, 0.000000, 0.000000}
\pgfsetstrokecolor{dialinecolor}
\draw (23.900000\du,31.000000\du)--(23.900000\du,32.000000\du)--(27.000000\du,32.000000\du)--(27.000000\du,18.000000\du)--(25.400000\du,18.000000\du);
}}
\pgfsetlinewidth{0.100000\du}
\pgfsetdash{}{0pt}
\pgfsetdash{}{0pt}
\pgfsetmiterjoin
\pgfsetbuttcap
{
\definecolor{dialinecolor}{rgb}{0.000000, 0.000000, 0.000000}
\pgfsetfillcolor{dialinecolor}
% was here!!!
\pgfsetarrowsend{latex}
{\pgfsetcornersarced{\pgfpoint{0.000000\du}{0.000000\du}}\definecolor{dialinecolor}{rgb}{0.000000, 0.000000, 0.000000}
\pgfsetstrokecolor{dialinecolor}
\draw (10.900000\du,31.000000\du)--(10.900000\du,32.000000\du)--(8.000000\du,32.000000\du)--(8.000000\du,18.000000\du)--(9.500000\du,18.000000\du);
}}
\pgfsetlinewidth{0.100000\du}
\pgfsetdash{}{0pt}
\pgfsetdash{}{0pt}
\pgfsetbuttcap
{
\definecolor{dialinecolor}{rgb}{0.000000, 0.000000, 0.000000}
\pgfsetfillcolor{dialinecolor}
% was here!!!
\definecolor{dialinecolor}{rgb}{0.000000, 0.000000, 0.000000}
\pgfsetstrokecolor{dialinecolor}
\pgfpathmoveto{\pgfpoint{27.400000\du}{22.000000\du}}
\pgfpatharc{360}{181}{0.400000\du and 0.400000\du}
\pgfusepath{stroke}
}
\pgfsetlinewidth{0.100000\du}
\pgfsetdash{}{0pt}
\pgfsetdash{}{0pt}
\pgfsetmiterjoin
\pgfsetbuttcap
{
\definecolor{dialinecolor}{rgb}{0.000000, 0.000000, 0.000000}
\pgfsetfillcolor{dialinecolor}
% was here!!!
{\pgfsetcornersarced{\pgfpoint{0.000000\du}{0.000000\du}}\definecolor{dialinecolor}{rgb}{0.000000, 0.000000, 0.000000}
\pgfsetstrokecolor{dialinecolor}
\draw (29.000000\du,11.000000\du)--(29.000000\du,22.000000\du)--(27.400000\du,22.000000\du)--(27.400000\du,22.000000\du);
}}
\pgfsetlinewidth{0.100000\du}
\pgfsetdash{}{0pt}
\pgfsetdash{}{0pt}
\pgfsetbuttcap
{
\definecolor{dialinecolor}{rgb}{0.000000, 0.000000, 0.000000}
\pgfsetfillcolor{dialinecolor}
% was here!!!
\pgfsetarrowsend{latex}
\definecolor{dialinecolor}{rgb}{0.000000, 0.000000, 0.000000}
\pgfsetstrokecolor{dialinecolor}
\draw (26.600000\du,22.000000\du)--(24.600000\du,22.000000\du);
}
\pgfsetlinewidth{0.100000\du}
\pgfsetdash{}{0pt}
\pgfsetdash{}{0pt}
\pgfsetbuttcap
{
\definecolor{dialinecolor}{rgb}{0.000000, 0.000000, 0.000000}
\pgfsetfillcolor{dialinecolor}
% was here!!!
\definecolor{dialinecolor}{rgb}{0.000000, 0.000000, 0.000000}
\pgfsetstrokecolor{dialinecolor}
\pgfpathmoveto{\pgfpoint{29.400000\du}{14.800000\du}}
\pgfpatharc{360}{181}{0.400000\du and 0.400000\du}
\pgfusepath{stroke}
}
\pgfsetlinewidth{0.100000\du}
\pgfsetdash{}{0pt}
\pgfsetdash{}{0pt}
\pgfsetbuttcap
{
\definecolor{dialinecolor}{rgb}{0.000000, 0.000000, 0.000000}
\pgfsetfillcolor{dialinecolor}
% was here!!!
\definecolor{dialinecolor}{rgb}{0.000000, 0.000000, 0.000000}
\pgfsetstrokecolor{dialinecolor}
\draw (24.500000\du,14.750000\du)--(28.600000\du,14.800000\du);
}
\pgfsetlinewidth{0.100000\du}
\pgfsetdash{}{0pt}
\pgfsetdash{}{0pt}
\pgfsetbuttcap
\pgfsetmiterjoin
\pgfsetlinewidth{0.100000\du}
\pgfsetbuttcap
\pgfsetmiterjoin
\pgfsetdash{}{0pt}
\definecolor{dialinecolor}{rgb}{1.000000, 1.000000, 1.000000}
\pgfsetfillcolor{dialinecolor}
\pgfpathellipse{\pgfpoint{38.950000\du}{14.790000\du}}{\pgfpoint{0.750000\du}{0\du}}{\pgfpoint{0\du}{0.750000\du}}
\pgfusepath{fill}
\definecolor{dialinecolor}{rgb}{0.000000, 0.000000, 0.000000}
\pgfsetstrokecolor{dialinecolor}
\pgfpathellipse{\pgfpoint{38.950000\du}{14.790000\du}}{\pgfpoint{0.750000\du}{0\du}}{\pgfpoint{0\du}{0.750000\du}}
\pgfusepath{stroke}
\pgfsetbuttcap
\pgfsetmiterjoin
\pgfsetdash{}{0pt}
\definecolor{dialinecolor}{rgb}{0.000000, 0.000000, 0.000000}
\pgfsetstrokecolor{dialinecolor}
\draw (38.950000\du,14.040000\du)--(38.950000\du,15.540000\du);
\pgfsetbuttcap
\pgfsetmiterjoin
\pgfsetdash{}{0pt}
\definecolor{dialinecolor}{rgb}{0.000000, 0.000000, 0.000000}
\pgfsetstrokecolor{dialinecolor}
\draw (38.200000\du,14.790000\du)--(39.700000\du,14.790000\du);
\pgfsetlinewidth{0.100000\du}
\pgfsetdash{}{0pt}
\pgfsetdash{}{0pt}
\pgfsetbuttcap
{
\definecolor{dialinecolor}{rgb}{0.000000, 0.000000, 0.000000}
\pgfsetfillcolor{dialinecolor}
% was here!!!
\pgfsetarrowsend{latex}
\definecolor{dialinecolor}{rgb}{0.000000, 0.000000, 0.000000}
\pgfsetstrokecolor{dialinecolor}
\draw (39.000000\du,11.000000\du)--(38.962036\du,13.991553\du);
}
\pgfsetlinewidth{0.100000\du}
\pgfsetdash{}{0pt}
\pgfsetdash{}{0pt}
\pgfsetbuttcap
{
\definecolor{dialinecolor}{rgb}{0.000000, 0.000000, 0.000000}
\pgfsetfillcolor{dialinecolor}
% was here!!!
\pgfsetarrowsend{latex}
\definecolor{dialinecolor}{rgb}{0.000000, 0.000000, 0.000000}
\pgfsetstrokecolor{dialinecolor}
\draw (29.400000\du,14.800000\du)--(38.200000\du,14.790000\du);
}
\pgfsetlinewidth{0.100000\du}
\pgfsetdash{}{0pt}
\pgfsetdash{}{0pt}
\pgfsetbuttcap
{
\definecolor{dialinecolor}{rgb}{0.000000, 0.000000, 0.000000}
\pgfsetfillcolor{dialinecolor}
% was here!!!
\pgfsetarrowsend{latex}
\definecolor{dialinecolor}{rgb}{0.000000, 0.000000, 0.000000}
\pgfsetstrokecolor{dialinecolor}
\draw (39.700000\du,14.790000\du)--(43.800000\du,14.800000\du);
}
% setfont left to latex
\definecolor{dialinecolor}{rgb}{0.000000, 0.000000, 0.000000}
\pgfsetstrokecolor{dialinecolor}
\node[anchor=west] at (43.800000\du,14.000000\du){$Z$};
% setfont left to latex
\definecolor{dialinecolor}{rgb}{0.000000, 0.000000, 0.000000}
\pgfsetstrokecolor{dialinecolor}
\node[anchor=west] at (30.000000\du,33.000000\du){F};
% setfont left to latex
\definecolor{dialinecolor}{rgb}{0.000000, 0.000000, 0.000000}
\pgfsetstrokecolor{dialinecolor}
\node[anchor=west] at (33.000000\du,14.000000\du){$W$};

\end{tikzpicture}

    \end{minipage}
\end{frame}


\begin{frame}{ZUC stream cipher}{Linear feedback shift register randomness}
    \begin{block}{}
        \begin{itemize}
            \item \textcolor{red}{All key and IV bits equal 1}
            \item \textcolor{blue}{All key and IV bits equal 0}
        \end{itemize}
    \end{block}
    \begin{figure}[htbp]
        \centering
        \includegraphics[scale=0.5]{correlation}
        \caption{Correlation of LFSR state before and after initialization}
        \label{fig:corr}
    \end{figure}
\end{frame}

\begin{frame}{ZUC stream cipher}{Keystream randomness}
    \begin{block}{}
        \begin{itemize}
            \item $55$ keystream samples with long bit series, each $65504$
                bits long;
            \item $163 / 189$ tests run;
            \item successfully passed $162$ tests.
        \end{itemize}
    \end{block}
    \begin{figure}[htbp]
        \centering
        \includegraphics[scale=0.5]{stats}
        \caption{Keystream randomness testing results}
        \label{fig:corr}
    \end{figure}
\end{frame}

\begin{frame}{ZUC stream cipher}{Analysis results}
    \begin{block}{Summary}
        \begin{itemize}
            \item a defect in ZUC linear transformation is revealed;
            \item cipher structure does not allow to exploit the property;
            \item yet linear transformation needs improvement before being used
                for development of future cryptographic algorithms;
            \item cross correlation deviations of LFSR states are observed for
                the keys with long bit series;
            \item keystream satisfies randomness criteria according to NIST~STS.
        \end{itemize}
    \end{block}
    \begin{block}{Successful cryptographic design decisions}
        \begin{itemize}
            \item combining XOR and modulo addition;
            \item mixing finite state machine registers state;
            \item summation of finite state machine output with LFSR bits.
        \end{itemize}
    \end{block}
\end{frame}


\section{Perspective lightweight cipher properties and requirements}
\begin{frame}{Lightweight cryptography}{PRESENT cipher}
    \begin{minipage}[t]{0.65\textwidth}
        \begin{itemize}
            \item 64 bits encryption block.
            \item Substitution layer: S-box $F_2^4 \rightarrow F_2^4$. \\
            \item Permutation layer:
                \begin{equation}
                    \label{eqn:player}
                    \nonumber
                    P(i) = \left\{
                    \begin{array}{ll}
                        i \cdot 16 \mod 63, & i \in {0, \hdots, 62} \\
                        63,  & i = 63 \enspace .
                    \end{array} \right.
                \end{equation}
            \item 31 round + key whitening.
            \item Master-key 80 or 128 bits long.
            \item Key development:
                \begin{enumerate}
                    \setlength{\itemsep}{1pt}
                        \setlength{\parskip}{0pt}
                        \setlength{\parsep}{0pt}
                    \item $ [k_{79} k_{78} \hdots k_1 k_0] = [k_{18} k_{17} \hdots k_{20} k_{19}] $;
                    \item $ [K_{79} k_{78} k_{77} k_{76}] = S[k_{79} k_{78} k_{77} k_{76}] $;
                    \item $ [k_{19} k_{18} k_{17} k_{16} k_{15}] = [k_{19} k_{18} k_{17} k_{16} k_{15}
                        \oplus \mbox{round\_counter}] $.
                \end{enumerate}
        \end{itemize}
    \end{minipage}%
    \begin{minipage}[t]{0.35\linewidth}
        \begin{figure}[h]
            \centering
            \includegraphics[scale=0.2]{present}
            \label{fig:present}
        \end{figure}
    \end{minipage}
\end{frame}

\begin{frame}{Lightweight cryptography}{Ciphers comparison}
    \begin{block}{Rationale}
        \begin{itemize}
            \item PRESENT is designed for lightweight crypto; \\
                it is now an international standard: ISO/IEC~29192-2:2012;
            \item GOST is an encryption standard in CIS countries, \\
                has been exposed to cryptanalysis for over 20 years;
            \item AES is the most widely used cipher.
        \end{itemize}
    \end{block}
    \begin{table}[htbp]
        \centering
        \caption{Comparison of PRESENT, AES and GOST 28147-89}
        \label{tbl:comparison}
        \begin{tabular}{|l|p{1cm}|p{1cm}|p{2cm}|p{1.2cm}|p{1.5cm}|}
            \hline
            Cipher  & key, bits & block, bits & Throughput, Kb/s & Area, GE & Efficiency, $\frac{bps}{\text{GE}}$ \\
            \hline
            PRESENT & 80  & 64  & 11.7 & 1075 & 10.89 \\
            \hline
            GOST    & 256 & 64  & 14   & 800  & 17.5  \\
            \hline
            AES     & 128 & 128 & 80   & 3100 & 25.81 \\
            \hline
        \end{tabular}
    \end{table}
\end{frame}

\section{Algebraic cryptanalysis of GOST~28147}
\begin{frame}{Algebraic cryptanalysis}
    \begin{block}{Claude Shannon}
        ``Breaking a good cipher should require as much work as solving a system of
        simultaneous equations in a large number of unknowns of a complex type.''
    \end{block}
    \begin{minipage}[t]{0.5\textwidth}
        \begin{block}{Basic ideas}
            \begin{enumerate}
                \item cryptoalgorithm is described by a multivariate quadratic equation system;
                \item given plaintexts and ciphertexts the polynomial system is
                    solved to get key variables.
            \end{enumerate}
        \end{block}
    \end{minipage}%
    \hspace*{1ex}
    \begin{minipage}[t]{0.5\textwidth}
        \begin{figure}[htbp]
            \centering
            \input{gost}
            \caption{GOST~28147-89 Round Function}
        \end{figure}
    \end{minipage}%
\end{frame}

\begin{frame}[fragile, shrink=2]{Algebraic cryptanalysis}{Solving equation systems}
    \begin{block}{GOST~28147-89 equation system}
        \begin{itemize}
            \item implemented using SAGE symbolic algebra system;
            \item single round contains $325$ quadratic equations;
            \item full cipher is defined by $10432$ polynomials in $4416$ variables;
            \item $5$ rounds GOST polynomial system is solved using CryptoMiniSat.
        \end{itemize}
    \end{block}

    \begin{block}{CryptoMiniSat solver usage}
    \begin{enumerate}
        \item multivariate equation system is constructed in ANF using SAGE;
        \item polynomial system is converted from ANF to CNF with \verb+anf2cnf+;
        \item \verb+CryptoMiniSat+ finds satisfiable set of variable values.
    \end{enumerate}
    \end{block}
    \begin{example}
\begin{lstlisting}
sage: gost = Gost(rounds=5)
sage: f = gost.polynomial_system()
sage: solver = ANFSatSolver(f.ring())
sage: s, t = solver(f)
\end{lstlisting}
    \end{example}
\end{frame}

\begin{frame}{Conclusions}
    \small
    \begin{block}{ZUC cipher analysis}
        \begin{itemize}
            \item cryptographic properties of the cipher are researched and
                linear transformation defects are considered;
            \item cipher analysis did not reveal any practical
                weaknesses in the algorithm.
        \end{itemize}
    \end{block}

    \begin{block}{Lightweight crypto}
        \begin{itemize}
            \item while being a legacy cipher for data security GOST~28147 fits
                well for lightweight cryptography purposes (even better than PRESENT);
            \item it's possible to implement GOST cipher using only 800 GE.
        \end{itemize}
    \end{block}

    \begin{block}{Algebraic analysis}
        \begin{itemize}
            \item GOST algorithm is efficiently defined with quadratic equations;
            \item 5 rounds of GOST equation system are solvable at the moment;
            \item with further optimizations multivariate equation systems with
                more rounds may be successfully solved.
        \end{itemize}
    \end{block}
\end{frame}

\begin{frame}[allowframebreaks]{Publications}
    \scriptsize
    \begingroup
    \renewcommand{\chapter}[2]{}%
    \begin{thebibliography}{1}
            \providecommand*{\BibEmph}[1]{#1}
            \providecommand*{\cyrdash}{\hbox to.8em{--\hss--}}
            \providecommand*{\BibDash}{\ifdim\lastskip>0pt\unskip\nobreak\hskip.2em\fi\cyrdash\hskip.2em\ignorespaces}

            \bibitem{Kiyanchuk:DESSERT:2012}
            \BibEmph{Oliynykov~R.~V., Kiyanchuk~R.~I.}
            \newblock {Perspective Symmetric Block Cipher optimized for Hardware Implementation}~//
            \newblock {6-th International Conference ``Dependable Systems, Services \& Technologies (DESSERT'12)''}. \BibDash 2012.

            \bibitem{Kiyanchuk:visnyk:2012}
            \BibEmph{Kiyanchuk~R.~I., Oliynykov~R.~V.} \newblock {Linear transformation properties of
            ZUC cipher}~// \newblock \BibEmph{Visnyk}. \BibDash 2012. \BibDash{Mathematical modeling. Information technologies. Computer-aided
            control systems.}

            \bibitem{karazina:zuc}
            \BibEmph{{Kiyanchuk, R. I. and Oliynykov R. V.}} \newblock {Linear transformation
            properties of ZUC cipher}~// \newblock {Computer modeling in high-end technologies}~/
            Kharkiv national university of radio electronics. \BibDash {Kharkiv}, 2012. \BibDash P.~199 -- 202.

            \bibitem{Kiyanchuk:2012:Banking}
            \BibEmph{Kiyanchuk~R.~I.} \newblock {Differential analysis of S-functions
            [In~Russian]}~// \newblock {Scientific youth researching for European integration}~/
            Kharkiv university of banking. \BibDash Kharkiv, 2012. \BibDash Electronic resource on CD-ROM.

            \bibitem{Kiyanchuk:2012:MMF}
            \BibEmph{Kiyanchuk~R.~I.} \newblock {Differential analysis of S-functions
            [In~Russian]}~// \newblock {Radioelectronics and youth in XXI century}~/ Kharkiv
            national university of radio electronics. \BibDash Kharkiv, 2012. \BibDash {С.}~130 -- 131.

            \bibitem{Kiyanchuk:2011:MMF}
            \BibEmph{Kiyanchuk~R.~I.} \newblock {Comparative analysis of IDEA-like Block Symmetric
            Ciphers [In~Ukrainian]}~// \newblock {International Conference ``Computer
            Engineering''}~/ {Kharkiv National University of Radio Electronics} \BibDash Kharkiv, 2011. \BibDash April. \BibDash {С.}~225 -- 227.

            \bibitem{Kiyanchuk:IREF:2011:present}
            \BibEmph{Oliynykov~R.~V., Kiyanchuk~R.~I.} \newblock{Perspective Symmetric Block Cipher
            Optimized for Hardware Implementation [In~Russian]}~// \newblock{``Telecommunication
            Systems and Technologies''}~/ {Kharkiv National University of Radio
            Electronics}. \BibDash Vol.~II. \BibDash {Kharkiv, Ukraine}, 2011. \BibDash October. \BibDash
            P.~321 -- 330.

            \bibitem{Kiyanchuk:2011:Customs}
            \BibEmph{Oliynykov~R.~V., Kiyanchuk~R.~I.} \newblock{Usage of T-functions in Symmetric
            Cryptographic Transformations [In~Russian]}~// \newblock{``Perspectives of Information
            and transport-customs technologies in customs affairs, external economic
            industry and organizations management''}~/ {Kharkiv National University of
            Radio Electronics}. \BibDash Dnipropetrovs'k, 2011. \BibDash December. \BibDash
            {С.}~213 -- 215. \BibDash Section 2.

            \bibitem{Kiyanchuk:2009:rijndael}
            \BibEmph{Dolgov~V.~I., Lysytska~I.~V., Kiyanchuk~R.~I.} \newblock {RIJNDAEL -- Is This a
            New or Well Forgotten Old Solution? [In~Russian]}~// \newblock{Computer Science and
            Technologies}~/ {Kharkiv National University of Radio Electronics}. \BibDash
            2009. \BibDash P.~32 -- 35.
    \end{thebibliography}
\endgroup
\end{frame}
\end{document}

